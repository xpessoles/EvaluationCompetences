\documentclass[10pt,fleqn]{article} % Default font size and left-justified equations
\usepackage[%
    pdftitle={Modélisation systèmes multiphysiques : Modélisation par fonction de transfert et schéma-blocs},
    pdfauthor={Xavier Pessoles}]{hyperref}
\input{style/new_style}
\input{style/macros_SII}
\def\classe{\textsf{PSI$\star$}}
\def\xxnumpartie{}
\def\xxpartie{}%Modéliser et corriger le comportement linéaire et non linéaire des systèmes multiphysiques}
\def\xxnumchapitre{Devoir surveillé de SII 3\vspace{.2cm}}
\def\xxchapitre{\hspace{.12cm}  }
\def\discipline{Sciences \\Industrielles de \\ l'Ingénieur}
\def\xxtete{Sciences Industrielles de l'Ingénieur}
\def\xxactivite{\textsf{DS 3}}
\def\xxonglet{\textsf{DS 3}}
\def\xxauteur{\textsl{Xavier Pessoles}}

\def\xxtitreexo{\noindent Table Azalée}
\def\xxsourceexo{\hspace{.2cm} Banque PT -- SIA 2019}


\def\xxfigures{\includegraphics[width=.6\linewidth]{fig_00}}
\def\xxpied{%
\textsf{DS 3}%dans le but de déterminer les contraintes géométriques dans les mécanismes\\% afin de valider leurs performances.\\
%Révisions 1 -- 2 -- 3 -- \xxactivite%
}

\def\xxposongletx{2}
\def\xxposonglettext{1.45}
\def\xxposonglety{20}
%\def\xxonglet{Part. 1 -- Ch. 3}


\usepackage{multicol}
\usepackage{siunitx}

\fichetrue
%\fichefalse

\proftrue
\proffalse

\tdtrue
\tdfalse

\courstrue
\coursfalse

\setcounter{secnumdepth}{5}
%---------------------------------------------------------------------------


\begin{document}
%\chapterimage{png/Fond_Cin}
\input{style/new_pagegarde}
%\vspace{4.5cm}
\pagestyle{fancy}
\thispagestyle{plain}


\def\columnseprulecolor{\color{ocre}}
\setlength{\columnseprule}{0.4pt} 


\begin{tikzpicture}[remember picture,overlay]
\draw[anchor=west] (-2cm,-4cm) node {\huge\sffamily\bfseries\color{black} %
\begin{minipage}{5cm}
\begin{center}
\LARGE\sffamily\color{ocre}\textbf{\textsc{\xxactivite}}

\begin{center}
\xxfigures
\end{center}

\end{center}
\end{minipage} \hfill
\begin{minipage}[c]{12cm}
\begin{titrechapitre}
\renewcommand{\baselinestretch}{1.1} 
\large\sffamily\textbf{\textsc{\xxtitreexo}}

\small\sffamily{\textbf{\textit{\color{black!70}\xxsourceexo}}}
\vspace{.5cm}

\renewcommand{\baselinestretch}{1} 
\normalsize\normalfont
%\xxcompetences
\end{titrechapitre}
\end{minipage}};
\end{tikzpicture}

\vspace{6cm}


\textbf{Commentaires généraux}
Ce DS est plutôt réussi en moyenne. Beaucoup ont progressé en liaisons équivalentes, hyperstatisme, diagramme de Bode. Il faut encore progresser en cinématique et en dynamique (en calcul et stratégie de calcul).





 \textbf{Extraits du rapport de jury pouvant être appliqués à l'ensemble de la classe}

\textbf{Première partie}
\begin{itemize}
\item confusion entre temps disponible après traitement et temps de traitement total due à une lecture
trop rapide de la question 1 ;
\item manipulation erronée d’unités ou de relations géométriques simples.
\end{itemize}

\textbf{Deuxième partie}
Les réponses à la question Q5 ont
été la plupart du temps trop approximatives et trop souvent exprimées avec un langage technique
insuffisamment précis (par exemple incliner, pencher, etc., pour désigner une rotation).


La Q11 portant spécifiquement sur la modélisation dynamique a posé des difficultés importantes à une grande partie des candidats. En n’appliquant pas
avec suffisamment de rigueur les étapes attendues, (choix du (ou des) solide(s) isolé(s), bilan des
actions mécaniques extérieures, choix du théorème et des équations utiles) beaucoup de candidats
abordent les calculs d’une manière désordonnée et souvent confuse qui ne leur permet pas d’aboutir
au modèle recherché. Les lacunes dans la maitrise des méthodes de calcul des moments cinétiques
et dynamiques ajoutent une difficulté supplémentaire à la conduite de ces analyses.

La question Q12 demande d’émettre des hypothèses sur la simplification du modèle. La réponse à
cette question, issue directement de l’analyse du modèle déterminée à la question Q11, a été bien
réussie par les candidats ayant pu mener à bien la démarche de modélisation.



\textbf{Troisième partie}

La démarche de développement des modèles dynamiques simplifiés Q13-14 est semblable à celle
des questions Q10-11 mais sur une chaine cinématique plus simple. Cela a permis une meilleure
réussite -- \textit{au concours, pas dans notre classe :)}.

Le jury a apprécié que les meilleurs candidats, en s’appuyant sur une approche plus
structurée, proposent un traitement plus efficace. Malgré la réussite à ces questions, le jury note
que pour beaucoup la dynamique reste un problème difficile avec des réponses confuses.

La question Q15 extrêmement simple où il s’agit de donner la constante de temps d’une fonction du
premier ordre (de pulsation de coupure connue) surprend par le nombre d’erreurs : méconnaissance
du cours, ignorance des unités, confusion pulsation/fréquence, etc.

La démarche de justification des besoins de la loi de commande, abordée dans les questions Q16-17,
ne présente pas de difficultés importantes. Elle a été relativement bien traitée par une partie
importante des candidats. Il faut noter toutefois que beaucoup ne prennent pas le temps de lire
le sujet : la fonction de transfert, à la question Q16, est demandée sous la forme canonique. De
plus, dans la question Q17 beaucoup se trompent dans la lecture des abaques donnés sous forme
semi-logarithmique ou omettent de préciser les unités des résultats numériques obtenus.


L’utilisation des diagrammes asymptotiques de Bode semble assez bien maitrisée. Cependant trop
d’erreurs pénalisent une partie non négligeable de candidats en raison d’une démarche désordonnée.
Dans le cas de ce sujet, question Q18, il suffit de réécrire la fonction de transfert en boucle
ouverte de la chaine d’asservissement, composée uniquement de fonctions du premier ordre avec
un intégrateur, de noter les pulsations de brisure dues aux différents pôles/zéros et de rapporter
ces valeurs aux variations de phase de $\pm$ 90° dans le cas du diagramme asymptotique. Les erreurs
principales sont dues à un manque de méthode, surprenant dans un cas simple, et à quelques
erreurs de lecture d’une échelle semi-logarithmique.

La synthèse du régulateur, objet des questions Q19-20, est peu guidée. La connaissance seule
de résultats ponctuels de cours est insuffisante pour traiter ces questions car il est nécessaire
de développer une démarche de synthèse en s’appuyant sur les méthodes vues en cours. Peu de
candidats ont pu mettre en place cette démarche d’une façon structurée. Pour obtenir les valeurs
numériques, la plupart des copies ayant abordé correctement ces questions ont utilisé une résolution
numérique (à la calculatrice) après la reformulation du problème de synthèse comme une équation
non linéaire à résoudre. Une formulation simple, avec peu de calculs pouvait être obtenue. Elle a
été faite par un nombre très réduit de candidats.

Enfin la validation de la démarche, l’analyse des résultats et des performances fait l’objet des
questions Q21-23. La question Q21, abordée par un nombre important de candidats, doit leur
permettre de montrer leur capacité d’analyse et de synthèse des résultats. Malheureusement, des
erreurs de lecture (confusion secondes et millisecondes), des confusions entre temps de réponse à 5%
et temps où la grandeur régulée atteint 95\% de la valeur finale ou l’incompréhension de la nature
de la réalité physique de la saturation mettent en évidence une absence de recul (sur un cas simple)
d’un nombre important de candidats. Au travers de leurs réponses, il apparait que la saturation est
un élément faisant partie du régulateur, que sa présence est justifiée par les performances obtenues
et non par les écarts importants observés dans les évolutions de la grandeur régulée pour les deux
cas simulés.

La première partie de la question de synthèse Q24 a été dans l’ensemble assez bien abordée même si
les réponses sont parfois confuses ou manquent d’argumentation précise. Les candidats n’appuient
pas suffisamment leurs réponses sur les données du sujet : cahier des charges, performances obtenues
comparées à celles espérées, etc. La deuxième partie demande de bien analyser les performances
demandées par le cahier des charges et de les mettre en relation avec la persistance rétinienne.
Cette question est difficile. Elle a été très peu abordée, voire non comprise.

\begin{minipage}[c]{.45\linewidth} 
\Large \textbf{\textsf{NOM19 Prenom19}} 
 
 \normalsize Note harmonisée 6.11/20 
 
Rang 6
 
Moyenne classe harmonisée 7.13/20 
 
Moyenne question traitées 6.82/20 
 
Rang question traitées 3 
 
Commentaires : 
Comment20 
\end{minipage}\hfill 
\begin{minipage}[c]{.45\linewidth}  
\begin{center}
\includegraphics[width=.8\linewidth]{../histo.pdf} 
\end{center}
\end{minipage}
\footnotesize 
\begin{center} 
\begin{tabular}{|c|c|m{1cm}|c||c|c|m{1cm}|c||c|c|m{1cm}|c||c|c|m{1cm}|c|} 
\hline \textbf{Qu} & \textbf{Coef} & \textbf{Comp} & \textbf{/5} & \textbf{Qu} & \textbf{Coef} & \textbf{Comp} & \textbf{/5} & \textbf{Qu} & \textbf{Coef} & \textbf{Comp} & \textbf{/5} & \textbf{Qu} & \textbf{Coef} & \textbf{Comp} & \textbf{/5} \\ 
\hline 
\hline 
Q1 & 5 & A1-01 & NT & Q1 & 5 & A1-02 & 2 & Q2 & 5 & A1-03 & 5 & Q2 & 5 & A1-04 & 0 \\ \hline 
 
Q3 & 5 & A1-05 & 1 & Q4 & 5 & A2-01 & 4 & Q5 & 5 & A2-02 & NT & Q6 & 5 & A2-03 & 2 \\ \hline 
 
Q7 & 5 & A3-01 & 5 & Q8 & 5 & A3-02 & 0 & Q9 & 10 & A3-03 & 1 &  &  &  &  \\ \hline 
 
\end{tabular} 
\end{center} 
\normalsize 
 


%\section{A -- Analyser}  
\subsection{A1 -- Analyser le besoin et les exigences}  
\subsubsection*{A1-01 -- Décrire le besoin et les exigences.}  
\begin{center} 
\includegraphics{A1-01.pdf} 
\end{center} 
\subsubsection*{A1-02 -- Traduire un besoin fonctionnel en exigences.}  
\begin{center} 
\includegraphics{A1-02.pdf} 
\end{center} 
\subsubsection*{A1-03 -- Définir les domaines d’application et les critères technico-économiques et environnementaux.}  
\begin{center} 
\includegraphics{A1-03.pdf} 
\end{center} 
\subsubsection*{A1-04 -- Qualifier et quantifier les exigences.}  
\begin{center} 
\includegraphics{A1-04.pdf} 
\end{center} 
\subsubsection*{A1-05 -- Évaluer l’impact environnemental et sociétal.}  
\begin{center} 
\includegraphics{A1-05.pdf} 
\end{center} 
\subsection{A2 -- Définir les frontières de l'analyse}  
\subsubsection*{A2-01 -- Isoler un système et justifier l’isolement.}  
\begin{center} 
\includegraphics{A2-01.pdf} 
\end{center} 
\subsubsection*{A2-02 -- Définir les éléments influents du milieu extérieur. }  
\begin{center} 
\includegraphics{A2-02.pdf} 
\end{center} 
\subsubsection*{A2-03 -- Identifier la nature des flux échangés traversant la frontière d’étude.}  
\begin{center} 
\includegraphics{A2-03.pdf} 
\end{center} 
\subsection{A3 -- Analyser l'organisation fonctionnelle et structurelle}  
\subsubsection*{A3-01 -- Associer les fonctions aux constituants.}  
\begin{center} 
\includegraphics{A3-01.pdf} 
\end{center} 
\subsubsection*{A3-02 -- Justifier le choix des constituants dédiés aux fonctions d’un système.}  
\begin{center} 
\includegraphics{A3-02.pdf} 
\end{center} 
\subsubsection*{A3-03 -- Identifier et décrire les chaines fonctionnelles du système.}  
\begin{center} 
\includegraphics{A3-03.pdf} 
\end{center} 
\subsubsection*{A3-04 -- Identifier et décrire les liens entre les chaines fonctionnelles.}  
\subsubsection*{A3-05 -- Caractériser un constituant de la chaine de puissance.}  
\subsubsection*{A3-06 -- Caractériser un constituant de la chaine d’information.}  
\subsubsection*{A3-07 -- Analyser un algorithme. }  
\subsubsection*{A3-08 -- Analyser les principes d'intelligence artificielle. }  
\subsubsection*{A3-09 -- Interpréter tout ou partie de l’évolution temporelle d’un système séquentiel.}  
\subsubsection*{A3-10 -- Identifier la structure d'un système asservi.}  
\subsection{A4 -- Analyser les performances et les écarts}  
\subsubsection*{A4-01 -- Extraire un indicateur de performance pertinent à partir du cahier des charges ou de résultats issus de l'expérimentation ou de la simulation.}  
\subsubsection*{A4-02 -- Caractériser les écarts entre les performances.}  
\subsubsection*{A4-03 -- Interpréter et vérifier la cohérence des résultats obtenus expérimentalement, analytiquement ou numériquement. }  
\subsubsection*{A4-04 -- Rechercher et proposer des causes aux écarts constatés.}  
\section{B -- Modéliser}  
\subsection{B1 -- Choisir les grandeurs physiques et les caractériser}  
\subsubsection*{B1-01 -- Identifier les performances à prévoir ou à évaluer.}  
\subsubsection*{B1-02 -- Identifier les grandeurs d'entrée et de sortie d’un modèle.}  
\subsubsection*{B1-03 -- Identifier les paramètres d’un modèle.}  
\subsubsection*{B1-04 -- Identifier et justifier les hypothèses nécessaires à la modélisation.}  
\subsection{B2 -- Proposer un modèle de connaissance et de comportement}  
\subsubsection*{B2-01 -- Choisir un modèle adapté aux performances à prévoir ou à évaluer.}  
\subsubsection*{B2-02 -- Compléter un modèle multiphysique.}  
\subsubsection*{B2-03 -- Associer un modèle aux composants des chaines fonctionnelles.}  
\subsubsection*{B2-04 -- Établir un modèle de connaissance par des fonctions de transfert.}  
\subsubsection*{B2-05 -- Modéliser le signal d'entrée.}  
\subsubsection*{B2-06 -- Établir un modèle de comportement à partir d'une réponse temporelle ou fréquentielle. }  
\subsubsection*{B2-07 -- Modéliser un système par schéma-blocs. }  
\subsubsection*{B2-08 -- Simplifier un modèle.}  
\subsubsection*{B2-09 -- Modéliser un correcteur numérique. }  
\subsubsection*{B2-10 -- Déterminer les caractéristiques d'un solide ou d'un ensemble de solides indéformables.}  
\subsubsection*{B2-11 -- Proposer une modélisation des liaisons avec leurs caractéristiques géométriques.}  
\subsubsection*{B2-12 -- Proposer un modèle cinématique à partir d'un système réel ou d'une maquette numérique.}  
\subsubsection*{B2-13 -- Modéliser la cinématique d'un ensemble de solides.}  
\subsubsection*{B2-14 -- Modéliser une action mécanique.}  
\subsubsection*{B2-15 -- Simplifier un modèle de mécanisme.}  
\subsubsection*{B2-16 -- Modifier un modèle pour le rendre isostatique.}  
\subsubsection*{B2-17 -- Décrire le comportement d'un système séquentiel.}  
\subsection{B3 -- Valider un modèle}  
\subsubsection*{B3-01 -- Vérifier la cohérence du modèle choisi en confrontant les résultats analytiques et/ou numériques aux résultats expérimentaux.}  
\subsubsection*{B3-02 -- Préciser les limites de validité d'un modèle.}  
\subsubsection*{B3-03 -- Modifier les paramètres et enrichir le modèle pour minimiser l’écart entre les résultats analytiques et/ou numériques et les résultats expérimentaux.}  
\section{C -- Résoudre}  
\subsection{C1 -- Proposer une démarche de résolution}  
\subsubsection*{C1-01 -- Proposer une démarche permettant d'évaluer les performances des systèmes asservis.}  
\subsubsection*{C1-02 -- Proposer une démarche de réglage d'un correcteur.}  
\subsubsection*{C1-03 -- Choisir une démarche de résolution d’un problème d'ingénierie numérique ou d'intelligence artificielle. }  
\subsubsection*{C1-04 -- Proposer une démarche permettant d'obtenir une loi entrée-sortie géométrique. }  
\subsubsection*{C1-05 -- Proposer une démarche permettant la détermination d’une action mécanique inconnue ou d'une loi de mouvement.}  
\subsection{C2 -- Mettre en œuvre une démarche de résolution analytique}  
\subsubsection*{C2-01 -- Déterminer la réponse temporelle.}  
\subsubsection*{C2-02 -- Déterminer la réponse fréquentielle. }  
\subsubsection*{C2-03 -- Déterminer les performances d'un système asservi.}  
\subsubsection*{C2-04 -- Mettre en œuvre une démarche de réglage d’un correcteur.}  
\subsubsection*{C2-05 -- Caractériser le mouvement d’un repère par rapport à un autre repère.}  
\subsubsection*{C2-06 -- Déterminer les relations entre les grandeurs géométriques ou cinématiques. }  
\subsubsection*{C2-07 -- Déterminer les actions mécaniques en statique.}  
\subsubsection*{C2-08 -- Déterminer les actions mécaniques en dynamique dans le cas où le mouvement est imposé.}  
\subsubsection*{C2-09 -- Déterminer la loi de mouvement dans le cas où les efforts extérieurs sont connus.}  
\subsection{C3 -- Mettre en œuvre une démarche de résolution numérique}  
\subsubsection*{C3-01 -- Mener une simulation numérique. }  
\subsubsection*{C3-02 -- Résoudre numériquement une équation ou un système d'équations. }  
\subsubsection*{C3-03 -- Résoudre un problème en utilisant une solution d'intelligence artificielle. }  
\section{D -- Expérimenter}  
\subsection{D1 -- Mettre en œuvre un système}  
\subsubsection*{D1-01 -- Mettre en œuvre un système en suivant un protocole.}  
\subsubsection*{D1-02 -- Repérer les constituants réalisant les principales fonctions des chaines fonctionnelles.}  
\subsubsection*{D1-03 -- Identifier les grandeurs physiques d’effort et de flux.}  
\subsection{D2 -- Proposer et justifier un protocole expérimental}  
\subsubsection*{D2-01 -- Choisir le protocole en fonction de l'objectif visé.}  
\subsubsection*{D2-02 -- Choisir les configurations matérielles et logicielles du système en fonction de l'objectif visé par l'expérimentation.}  
\subsubsection*{D2-03 -- Choisir les réglages du système en fonction de l'objectif visé par l'expérimentation.}  
\subsubsection*{D2-04 -- Choisir la grandeur physique à mesurer ou justifier son choix.}  
\subsubsection*{D2-05 -- Choisir les entrées à imposer et les sorties pour identifier un modèle de comportement.}  
\subsubsection*{D2-06 -- Justifier le choix d’un capteur ou d’un appareil de mesure vis-à-vis de la grandeur physique à mesurer.}  
\subsection{D3 -- Mettre en œuvre un protocole expérimental}  
\subsubsection*{D3-01 -- Régler les paramètres de fonctionnement d'un système.}  
\subsubsection*{D3-02 -- Mettre en œuvre un appareil de mesure adapté à la caractéristique de la grandeur à mesurer.}  
\subsubsection*{D3-03 -- Effectuer des traitements à partir de données. }  
\subsubsection*{D3-04 -- Identifier les erreurs de mesure.}  
\subsubsection*{D3-05 -- Identifier les erreurs de méthode.}  
\section{E -- Communiquer}  
\subsection{E1 -- Rechercher et traiter des informations}  
\subsubsection*{E1-01 -- Rechercher des informations.}  
\subsubsection*{E1-02 -- Distinguer les différents types de documents et de données en fonction de leurs usages.}  
\subsubsection*{E1-03 -- Vérifier la pertinence des informations (obtention, véracité, fiabilité et précision de l'information).}  
\subsubsection*{E1-04 -- Extraire les informations utiles d’un dossier technique.}  
\subsubsection*{E1-05 -- Lire et décoder un document technique.}  
\subsubsection*{E1-06 -- Trier les informations selon des critères.}  
\subsubsection*{E1-07 -- Effectuer une synthèse des informations disponibles dans un dossier technique.}  
\subsection{E2 -- Produire et échanger de l'information}  
\subsubsection*{E2-01 -- Choisir un outil de communication adapté à l’interlocuteur.}  
\subsubsection*{E2-02 -- Faire preuve d’écoute et confronter des points de vue.}  
\subsubsection*{E2-03 -- Présenter les étapes de son travail.}  
\subsubsection*{E2-04 -- Présenter de manière argumentée une synthèse des résultats.}  
\subsubsection*{E2-05 -- Produire des documents techniques adaptés à l'objectif de la communication. }  
\subsubsection*{E2-06 -- Utiliser un vocabulaire technique, des symboles et des unités adéquats.}  
\section{F -- Concevoir}  
\subsection{F1 -- Concevoir l'architecture d'un système innovant}  
\subsubsection*{F1-01 -- Proposer une architecture fonctionnelle et organique.}  
\subsection{F2 -- Proposer et choisir des solutions techniques}  
\subsubsection*{F2-01 -- Modifier la commande pour faire évoluer le comportement du système. }  


\end{document}
