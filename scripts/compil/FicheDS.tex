\documentclass[10pt,fleqn]{book} % Default font size and left-justified equations
\usepackage[%
    pdftitle={Modélisation systèmes multiphysiques : Modélisation par fonction de transfert et schéma-blocs},
    pdfauthor={Xavier Pessoles}]{hyperref}
\input{style/new_style}
%\input{../../../Style/macros_SII}
\input{../../../Style/environment_v2}
%\def\classe{\textsf{PSI $\star$}}
\def\classe{\textsf{PTSI}}
\def\xxnumpartie{}
\def\xxpartie{}%Modéliser et corriger le comportement linéaire et non linéaire des systèmes multiphysiques}
\def\xxnumchapitre{Devoir surveillé 2\vspace{.2cm}}
\def\xxchapitre{\hspace{.12cm}  }
\def\discipline{Informatique}
\def\xxtete{\discipline}
\def\xxactivite{\textsf{DS 2}}
\def\xxonglet{\textsf{\xxactivite}}
\def\xxauteur{\textsl{Xavier Pessoles}}

\def\xxtitreexo{\noindent  Représentation des nombres \& Images}
\def\xxsourceexo{\hspace{.2cm} Extrait de Banque PT 2021}


\def\xxfigures{}%\includegraphics[width=.5\linewidth]{fig_00}}
\def\xxpied{%
\textsf{\xxactivite}%dans le but de déterminer les contraintes géométriques dans les mécanismes\\% afin de valider leurs performances.\\
%Révisions 1 -- 2 -- 3 -- \xxactivite%
}

\def\xxposongletx{2}
\def\xxposonglettext{1.45}
\def\xxposonglety{20}
%\def\xxonglet{Part. 1 -- Ch. 3}


\usepackage{multicol}
\usepackage{siunitx}

\usepackage{tkz-kiviat}
\fichetrue
%\fichefalse


%\newcommand{\progress}[1]{
%    \begin{tikzpicture}
%        \fill[ocre!20, rounded corners =1 mm] (0,0) rectangle (#1/100*2.5,.4);
%         \draw[ocre,thick, rounded corners =1 mm] (0,0) rectangle (2.5,.4);
%         \node at (1.25,0.2) {\textcolor{black}{#1~\%}};
%    \end{tikzpicture}
%}

\proftrue
\proffalse

\tdtrue
\tdfalse

\courstrue
\coursfalse

\setcounter{secnumdepth}{5}
%---------------------------------------------------------------------------


\begin{document}
%\chapterimage{png/Fond_Cin}
\input{style/new_pagegarde}
%\vspace{4.5cm}
\pagestyle{fancy}
\thispagestyle{plain}


\def\columnseprulecolor{\color{ocre}}
\setlength{\columnseprule}{0pt} 


\begin{tikzpicture}[remember picture,overlay]
\draw[anchor=west] (-2cm,-4cm) node {\huge\sffamily\bfseries\color{black} %
\begin{minipage}{5cm}
\begin{center}
\LARGE\sffamily\color{ocre}\textbf{\textsc{\xxactivite}}

\begin{center}
\xxfigures
\end{center}

\end{center}
\end{minipage} \hfill
\begin{minipage}[c]{12cm}
\begin{titrechapitre}
\renewcommand{\baselinestretch}{1.1} 
\large\sffamily\textbf{\textsc{\xxtitreexo}}

\small\sffamily{\textbf{\textit{\color{black!70}\xxsourceexo}}}
\vspace{.5cm}

\renewcommand{\baselinestretch}{1} 
\normalsize\normalfont
%\xxcompetences
\end{titrechapitre}
\end{minipage}};
\end{tikzpicture}

\vspace{6cm}
%

\textbf{Remarques générales}
\begin{itemize}
\item Ensemble satisfaisant. Les fonctions de bases sont globalement bien codées. \hfill \progress{60}
\item Le cours sur les variants invariants complexité et plutôt bien appréhendé. \hfill \progress{60}
\item Les questions sur la bordure ont eu un succès très mitigé. Il s'agissait d'ajouter une bordure et pas de remplacer les pixels du bords par des pixels noirs. \hfill \progress{75}
\end{itemize}


\begin{minipage}[c]{.45\linewidth} 
\Large \textbf{\textsf{NOM19 Prenom19}} 
 
 \normalsize Note harmonisée 6.11/20 
 
Rang 6
 
Moyenne classe harmonisée 7.13/20 
 
Moyenne question traitées 6.82/20 
 
Rang question traitées 3 
 
Commentaires : 
Comment20 
\end{minipage}\hfill 
\begin{minipage}[c]{.45\linewidth}  
\begin{center}
\includegraphics[width=.8\linewidth]{../histo.pdf} 
\end{center}
\end{minipage}
\footnotesize 
\begin{center} 
\begin{tabular}{|c|c|m{1cm}|c||c|c|m{1cm}|c||c|c|m{1cm}|c||c|c|m{1cm}|c|} 
\hline \textbf{Qu} & \textbf{Coef} & \textbf{Comp} & \textbf{/5} & \textbf{Qu} & \textbf{Coef} & \textbf{Comp} & \textbf{/5} & \textbf{Qu} & \textbf{Coef} & \textbf{Comp} & \textbf{/5} & \textbf{Qu} & \textbf{Coef} & \textbf{Comp} & \textbf{/5} \\ 
\hline 
\hline 
Q1 & 5 & A1-01 & NT & Q1 & 5 & A1-02 & 2 & Q2 & 5 & A1-03 & 5 & Q2 & 5 & A1-04 & 0 \\ \hline 
 
Q3 & 5 & A1-05 & 1 & Q4 & 5 & A2-01 & 4 & Q5 & 5 & A2-02 & NT & Q6 & 5 & A2-03 & 2 \\ \hline 
 
Q7 & 5 & A3-01 & 5 & Q8 & 5 & A3-02 & 0 & Q9 & 10 & A3-03 & 1 &  &  &  &  \\ \hline 
 
\end{tabular} 
\end{center} 
\normalsize 
 


\newpage

%\begin{multicols}{2}
%\begin{tikzpicture}[scale=1, label distance = .5cm]
%	\tkzKiviatDiagram[lattice=5,gap=.3,space=.3,step=1,radial style/.style ={-latex},lattice style/.style ={blue!30}]%
%		{STAT-01,STAT-02,STAT-03,STAT-04,STAT-05,STAT-06}
%	\tkzKiviatLine[thick,color=red,
%		mark=ball,
%		ball color=red,
%		mark size=3pt,
%		fill=red!20](3,2,3,4,5,1)
%	\tkzKiviatLine[thick,color=blue,mark=ball,
%		mark size=3pt,fill=blue!20,	opacity=.5](3,3,3,3,3,3)
%\end{tikzpicture}
%\end{multicols}

\begin{minipage}[c]{.3\linewidth}

\begin{tikzpicture}[scale=1]
	\tkzKiviatDiagram[lattice=5,gap=.3,space=.3,step=1,radial style/.style ={-latex},lattice style/.style ={blue!30}]%
		{SYS-01,SYS-02,SYS-03,SYS-04,SYS-05,SYS-06}
	\tkzKiviatLine[thick,color=red,
		mark=ball,
		ball color=red,
		mark size=3pt,
		fill=red!20](3,2,3,4,5,1)
	\tkzKiviatLine[thick,color=blue,mark=ball,
		mark size=3pt,fill=blue!20,	opacity=.5](3,3,3,3,3,2)
\end{tikzpicture}
\end{minipage}\hfill
\begin{minipage}[c]{.6\linewidth}

\footnotesize
\allSysComp
\normalsize

\end{minipage}



\begin{minipage}[c]{.3\linewidth}
\begin{tikzpicture}[scale=1, label distance = .5cm]
	\tkzKiviatDiagram[lattice=5,gap=.3,space=.3,step=1,radial style/.style ={-latex},lattice style/.style ={blue!30}]%
		{STAT-01,STAT-02,STAT-03,STAT-04,STAT-05}
	\tkzKiviatLine[thick,color=red,
		mark=ball,
		ball color=red,
		mark size=3pt,
		fill=red!20](3,2,3,4,5)
	\tkzKiviatLine[thick,color=blue,mark=ball,
		mark size=3pt,fill=blue!20,	opacity=.5](3,3,3,3,3)
\end{tikzpicture}
\end{minipage}\hfill
\begin{minipage}[c]{.6\linewidth}


\allGeoComp

\end{minipage}

\begin{minipage}[c]{.3\linewidth}
\begin{tikzpicture}[scale=1, label distance = .5cm]
	\tkzKiviatDiagram[lattice=5,gap=.3,space=.3,step=1,radial style/.style ={-latex},lattice style/.style ={blue!30}]%
		{STAT-01,STAT-02,STAT-03,STAT-04,STAT-05}
	\tkzKiviatLine[thick,color=red,
		mark=ball,
		ball color=red,
		mark size=3pt,
		fill=red!20](3,2,3,4,5)
	\tkzKiviatLine[thick,color=blue,mark=ball,
		mark size=3pt,fill=blue!20,	opacity=.5](3,3,3,3,3)
\end{tikzpicture}
\end{minipage}\hfill
\begin{minipage}[c]{.6\linewidth}

\footnotesize
\allCinComp
\normalsize
\end{minipage}

\begin{minipage}[c]{.3\linewidth}
\begin{tikzpicture}[scale=1, label distance = .5cm]
	\tkzKiviatDiagram[lattice=5,gap=.3,space=.3,step=1,radial style/.style ={-latex},lattice style/.style ={blue!30}]%
		{STAT-01,STAT-02,STAT-03,STAT-04,STAT-05}
	\tkzKiviatLine[thick,color=red,
		mark=ball,
		ball color=red,
		mark size=3pt,
		fill=red!20](3,2,3,4,5)
	\tkzKiviatLine[thick,color=blue,mark=ball,
		mark size=3pt,fill=blue!20,	opacity=.5](3,3,3,3,3)
\end{tikzpicture}
\end{minipage}\hfill
\begin{minipage}[c]{.6\linewidth}

\footnotesize
\allStatComp
\normalsize
\end{minipage}


\begin{minipage}[c]{.3\linewidth}
\begin{tikzpicture}[scale=1, label distance = .5cm]
	\tkzKiviatDiagram[lattice=5,gap=.3,space=.3,step=1,radial style/.style ={-latex},lattice style/.style ={blue!30}]%
		{STAT-01,STAT-02,STAT-03,STAT-04,STAT-05}
	\tkzKiviatLine[thick,color=red,
		mark=ball,
		ball color=red,
		mark size=3pt,
		fill=red!20](3,2,3,4,5)
	\tkzKiviatLine[thick,color=blue,mark=ball,
		mark size=3pt,fill=blue!20,	opacity=.5](3,3,3,3,3)
\end{tikzpicture}
\end{minipage}\hfill
\begin{minipage}[c]{.6\linewidth}

\footnotesize
\allChsComp
\normalsize
\end{minipage}


\begin{minipage}[c]{.3\linewidth}
\begin{tikzpicture}[scale=1, label distance = .5cm]
	\tkzKiviatDiagram[lattice=5,gap=.3,space=.3,step=1,radial style/.style ={-latex},lattice style/.style ={blue!30}]%
		{STAT-01,STAT-02,STAT-03,STAT-04,STAT-05}
	\tkzKiviatLine[thick,color=red,
		mark=ball,
		ball color=red,
		mark size=3pt,
		fill=red!20](3,2,3,4,5)
	\tkzKiviatLine[thick,color=blue,mark=ball,
		mark size=3pt,fill=blue!20,	opacity=.5](3,3,3,3,3)
\end{tikzpicture}
\end{minipage}\hfill
\begin{minipage}[c]{.6\linewidth}

\footnotesize
\allDynComp
\normalsize
\end{minipage}


\begin{minipage}[c]{.3\linewidth}
\begin{tikzpicture}[scale=1, label distance = .5cm]
	\tkzKiviatDiagram[lattice=5,gap=.3,space=.3,step=1,radial style/.style ={-latex},lattice style/.style ={blue!30}]%
		{STAT-01,STAT-02,STAT-03,STAT-04,STAT-05}
	\tkzKiviatLine[thick,color=red,
		mark=ball,
		ball color=red,
		mark size=3pt,
		fill=red!20](3,2,3,4,5)
	\tkzKiviatLine[thick,color=blue,mark=ball,
		mark size=3pt,fill=blue!20,	opacity=.5](3,3,3,3,3)
\end{tikzpicture}
\end{minipage}\hfill
\begin{minipage}[c]{.6\linewidth}

\footnotesize
\allTecComp
\normalsize
\end{minipage}


\begin{minipage}[c]{.3\linewidth}
\begin{tikzpicture}[scale=1, label distance = .5cm]
	\tkzKiviatDiagram[lattice=5,gap=.3,space=.3,step=1,radial style/.style ={-latex},lattice style/.style ={blue!30}]%
		{STAT-01,STAT-02,STAT-03,STAT-04,STAT-05,STAT-06,STAT-07,STAT-08,STAT-09,STAT-10,STAT-11}
	\tkzKiviatLine[thick,color=red,
		mark=ball,
		ball color=red,
		mark size=3pt,
		fill=red!20](3,2,3,4,5,2,3,4,5,4,5)
	\tkzKiviatLine[thick,color=blue,mark=ball,
		mark size=3pt,fill=blue!20,	opacity=.5](3,3,3,3,3,2,3,4,5,4,5)
\end{tikzpicture}
\end{minipage}\hfill
\begin{minipage}[c]{.6\linewidth}

\footnotesize
\allSlciComp
\normalsize
\end{minipage}


\begin{minipage}[c]{.3\linewidth}
\begin{tikzpicture}[scale=1, label distance = .5cm]
	\tkzKiviatDiagram[lattice=5,gap=.3,space=.3,step=1,radial style/.style ={-latex},lattice style/.style ={blue!30}]%
		{STAT-01,STAT-02,STAT-03,STAT-04,STAT-05}
	\tkzKiviatLine[thick,color=red,
		mark=ball,
		ball color=red,
		mark size=3pt,
		fill=red!20](3,2,3,4,5)
	\tkzKiviatLine[thick,color=bleuxp,mark=ball,
		mark size=2pt,fill=bleuxp!20,	opacity=.5](3,3,3,3,3)
\end{tikzpicture}
\end{minipage}\hfill
\begin{minipage}[c]{.6\linewidth}

\footnotesize
\allPerfComp
\normalsize
\end{minipage}



\begin{minipage}[c]{.3\linewidth}
\begin{tikzpicture}[scale=1, label distance = .5cm]
	\tkzKiviatDiagram[lattice=5,gap=.3,space=.3,step=1,radial style/.style ={-latex},lattice style/.style ={blue!30}]%
		{STAT-01,STAT-02,STAT-03,STAT-04,STAT-05}
	\tkzKiviatLine[thick,color=red,
		mark=ball,
		ball color=red,
		mark size=3pt,
		fill=red!20](3,2,3,4,5)
	\tkzKiviatLine[thick,color=blue,mark=ball,
		mark size=3pt,fill=blue!20,	opacity=.5](3,3,3,3,3)
\end{tikzpicture}
\end{minipage}\hfill
\begin{minipage}[c]{.6\linewidth}


\allCorComp

\end{minipage}



\end{document}
