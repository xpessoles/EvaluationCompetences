\documentclass[10pt,fleqn]{article} % Default font size and left-justified equations
\usepackage[%
    pdftitle={Modélisation systèmes multiphysiques : Modélisation par fonction de transfert et schéma-blocs},
    pdfauthor={Xavier Pessoles}]{hyperref}
\input{style/new_style}
\input{style/macros_SII}
\def\classe{\textsf{PSI$\star$}}
\def\xxnumpartie{}
\def\xxpartie{}%Modéliser et corriger le comportement linéaire et non linéaire des systèmes multiphysiques}
\def\xxnumchapitre{Devoir surveillé de SII 3\vspace{.2cm}}
\def\xxchapitre{\hspace{.12cm}  }
\def\discipline{Sciences \\Industrielles de \\ l'Ingénieur}
\def\xxtete{Sciences Industrielles de l'Ingénieur}
\def\xxactivite{\textsf{DS 3}}
\def\xxonglet{\textsf{DS 3}}
\def\xxauteur{\textsl{Xavier Pessoles}}

\def\xxtitreexo{\noindent Table Azalée}
\def\xxsourceexo{\hspace{.2cm} Banque PT -- SIA 2019}


\def\xxfigures{\includegraphics[width=.6\linewidth]{fig_00}}
\def\xxpied{%
\textsf{DS 3}%dans le but de déterminer les contraintes géométriques dans les mécanismes\\% afin de valider leurs performances.\\
%Révisions 1 -- 2 -- 3 -- \xxactivite%
}

\def\xxposongletx{2}
\def\xxposonglettext{1.45}
\def\xxposonglety{20}
%\def\xxonglet{Part. 1 -- Ch. 3}


\usepackage{multicol}
\usepackage{siunitx}

\fichetrue
%\fichefalse

\proftrue
\proffalse

\tdtrue
\tdfalse

\courstrue
\coursfalse

\setcounter{secnumdepth}{5}
%---------------------------------------------------------------------------


\begin{document}
%\chapterimage{png/Fond_Cin}
\input{style/new_pagegarde}
%\vspace{4.5cm}
\pagestyle{fancy}
\thispagestyle{plain}


\def\columnseprulecolor{\color{ocre}}
\setlength{\columnseprule}{0.4pt} 


\begin{tikzpicture}[remember picture,overlay]
\draw[anchor=west] (-2cm,-4cm) node {\huge\sffamily\bfseries\color{black} %
\begin{minipage}{5cm}
\begin{center}
\LARGE\sffamily\color{ocre}\textbf{\textsc{\xxactivite}}

\begin{center}
\xxfigures
\end{center}

\end{center}
\end{minipage} \hfill
\begin{minipage}[c]{12cm}
\begin{titrechapitre}
\renewcommand{\baselinestretch}{1.1} 
\large\sffamily\textbf{\textsc{\xxtitreexo}}

\small\sffamily{\textbf{\textit{\color{black!70}\xxsourceexo}}}
\vspace{.5cm}

\renewcommand{\baselinestretch}{1} 
\normalsize\normalfont
%\xxcompetences
\end{titrechapitre}
\end{minipage}};
\end{tikzpicture}

\vspace{6cm}


\textbf{Remarques générales -- Rapport du jury}
\begin{itemize}
\item Q1 : première question assez élémentaire qui n’a pas souvent été correctement traitée. Audelà de confusion entre les sinus et cosinus, des résultats farfelus ont souvent été donnés.
\item Q2 : question assez simple quand la précédente a été traitée correctement. On notera que
des candidats ne réutilisent pas le résultat précédent et arrivent avec du bon sens à traiter
celle-ci.
\item Q3 : question assez bien traitée.
\item Q4 : question très souvent mal traitée faute d’utiliser une méthode correcte. On notera des
erreurs dans la formule de dérivation vectorielle (calculs et/ou point fixe mal choisi), ou bien
des formules de Varignon utilisées avec des points où les vitesses étaient non nulles comme
les candidats l’avançaient. De nombreuses projections inutiles ont aussi compliqué les
calculs et complexifié l’expression du résultat.
\item Q5 : question peu traitée et souvent fausse à cause du résultat précédent. De grandes
difficultés dans les conversions élémentaires tr/min en rad/s, degrés en radians… Bien que
l'énoncé le suggérait, rares sont les candidats qui ont cherché à apprécier l'importance
relative des différentes composantes de la vitesse.
\item Q6 : beaucoup de candidats utilisent la formule du corps 0 au lieu de prendre celle du bras 1.
\item Q6 et Q7 : pour les questions de dynamique, l’imagination des candidats pour produire des
formules non homogènes est quasiment sans limite. Les candidats devraient s’abstenir de
répondre plutôt que d’écrire n’importe quoi. De nombreuses copies ne contiennent que des
calculs illisibles sans aucune explication sur la démarche suivie et l’origine des
simplifications effectuées au cours du calcul.
\item Q8 : question souvent mal traitée quand la précédente l’a été correctement. Les candidats
jugeant souvent que le résultat se devait d’être identique à celui de la Q7.
\item Q9 : peu de candidats écrivent explicitement que le moment dynamique de l’ensemble est
la somme des moments dynamiques…
\item Q10 : question assez bien traitée quand elle a été abordée.
\item Q11 : beaucoup de candidats pensent obtenir une matrice diagonale sans aucune
justification.
\item Q12 : question numérique qui a été assez bien traitée si on omet les interprétations farfelues
sur les variations d’inertie.
\item Q13 : beaucoup de candidats n’ont pas compris que la portance étant toujours vers le haut
comme l’indiquait très clairement l’énoncé.
\item Q14 : question rarement traitée et plutôt mal comprise.
\item Q15 : question qui a semblé très difficile pour les candidats. Il s’agissait d’un simple
changement de point d’un torseur donné. Les calculs ont souvent été faux, la formule de
changement de point utilisée avec le vecteur vitesse de rotation !!!…
\item Q16 : les candidats tentent d’utiliser les solutions classiques en plaçant des sphériques mais
ils en placent souvent partout. D’autres remplacent toutes les liaisons pivots par des pivots
glissants… Il ne suffit pas d’ajouter des degrés de liberté. Une analyse des nouvelles
mobilités est nécessaire. Les correcteurs sont conscients que ce type de questions reste
difficile pour les candidats.
\item Q17 : question assez étonnamment non traitée par l’ensemble des candidats.
\item Q18 : un très grand nombre de candidats n’arrivent pas à conclure à cette question pourtant
élémentaire.
\item Q19 : question où l’on a vu assez peu d’estimations des valeurs numériques avec des
ordres de grandeur. Un nombre assez important estime qu’une longueur s’exprime en mm²,
ou encore juge inutile de donner la variation angulaire dans l’unité demandée par l’énoncé.
\item Q20 : question peu mais assez bien traitée. Pourtant il suffisait de lire l’intersection d’une
courbe avec l’axe y = 0 ! Environ un tiers des candidats n’a pas pu le faire !
\item Q21 : les candidats oublient trop souvent de regarder tous les critères possibles.
\item Q22 : question peu traitée.
\item Q23 : peu de candidats arrivent à justifier le choix d’un modèle d’ordre 1 pour identifier le
comportement du moteur (connaissance trop fragile du cours). Peu de candidats ont
compris que la consigne était exprimée en % (le texte était pourtant clair) et ils ont souvent
repris la valeur de l’axe des ordonnées en vitesse. Bien que la question n’était pas exprimée
sous la forme la plus classique, les correcteurs s’étonnent de la non réussite à cette
question.
\item Q24 : question assez bien traitée.
\item Q25 : question assez bien traitée. Beaucoup de candidats ne vont pas au bout de
l’application numérique, même d’un ordre de grandeur.
\item Q26 : la justification d’un modèle du second ordre est souvent partielle, mais l’identification
est globalement correcte.
\item Q27 : question assez mal traitée. Les candidats savent calculer la FTBO, mais ont du mal
à tracer le diagramme asymptotique en gain dans un premier temps car ils n’arrivent pas à
placer la valeur particulière avec le terme 150/p ! De plus, l’allure du diagramme réel est
4 / 4
souvent fausse, notamment au niveau de la résonance qu’il suffisait pourtant de recopier…
Le tracé de la phase est globalement correct.
\item Q28 : question mal traitée souvent à cause des tracés faux aux questions précédentes.
Cependant la définition des marges ne semble pas du tout être maîtrisée. Notamment sur
leurs signes. À rappeler que la représentation d’une marge par une double flèche est une
erreur. Trop de candidats relèvent à tort les marges sur le tracé asymptotique
\item Q29 : question peu traitée et rarement juste avec l’enchaînement des questions
précédentes.
\item Q30 : question étonnamment assez mal traitée ! Il s’agit d’une question de début de 1re
année que les candidats n’ont pas réussi à traiter correctement. Les correcteurs attendaient
au moins un résultat sous forme de fraction rationnelle en remplaçant les blocs par les
fonctions de transfert données dans l’énoncé.
\item Q31 : le théorème de la valeur finale a été très peu énoncé proprement et un résultat, parfois
juste, a simplement été donné.
\item Q32 : question assez rarement traitée. Lorsqu’elle est traitée, la mauvaise exigence du
cahier des charges est retenue rendant le raisonnement hors sujet.
\item Q33 : pour beaucoup de candidats le correcteur PI améliore la stabilité.
\item Q34 : la robustesse est assez mal mesurée par les candidats qui regardent plutôt la valeur
maximale de l’écartement à la position d’équilibre que la valeur au retour à l’équilibre.
Beaucoup oublient de vérifier l’ensemble des critères et se contentent de la précision ou de
la rapidité ou de la stabilité.
\item Q35 : question très peu traitée. Un peu trop de réponses sous forme « donc tout
fonctionne » sans aucune justification avec ce qui a été fait avant. Certes la place était
limitée sur le document réponse.
\end{itemize}

\begin{minipage}[c]{.45\linewidth} 
\Large \textbf{\textsf{NOM19 Prenom19}} 
 
 \normalsize Note harmonisée 6.11/20 
 
Rang 6
 
Moyenne classe harmonisée 7.13/20 
 
Moyenne question traitées 6.82/20 
 
Rang question traitées 3 
 
Commentaires : 
Comment20 
\end{minipage}\hfill 
\begin{minipage}[c]{.45\linewidth}  
\begin{center}
\includegraphics[width=.8\linewidth]{../histo.pdf} 
\end{center}
\end{minipage}
\footnotesize 
\begin{center} 
\begin{tabular}{|c|c|m{1cm}|c||c|c|m{1cm}|c||c|c|m{1cm}|c||c|c|m{1cm}|c|} 
\hline \textbf{Qu} & \textbf{Coef} & \textbf{Comp} & \textbf{/5} & \textbf{Qu} & \textbf{Coef} & \textbf{Comp} & \textbf{/5} & \textbf{Qu} & \textbf{Coef} & \textbf{Comp} & \textbf{/5} & \textbf{Qu} & \textbf{Coef} & \textbf{Comp} & \textbf{/5} \\ 
\hline 
\hline 
Q1 & 5 & A1-01 & NT & Q1 & 5 & A1-02 & 2 & Q2 & 5 & A1-03 & 5 & Q2 & 5 & A1-04 & 0 \\ \hline 
 
Q3 & 5 & A1-05 & 1 & Q4 & 5 & A2-01 & 4 & Q5 & 5 & A2-02 & NT & Q6 & 5 & A2-03 & 2 \\ \hline 
 
Q7 & 5 & A3-01 & 5 & Q8 & 5 & A3-02 & 0 & Q9 & 10 & A3-03 & 1 &  &  &  &  \\ \hline 
 
\end{tabular} 
\end{center} 
\normalsize 
 


%\section{A -- Analyser}  
\subsection{A1 -- Analyser le besoin et les exigences}  
\subsubsection*{A1-01 -- Décrire le besoin et les exigences.}  
\begin{center} 
\includegraphics{A1-01.pdf} 
\end{center} 
\subsubsection*{A1-02 -- Traduire un besoin fonctionnel en exigences.}  
\begin{center} 
\includegraphics{A1-02.pdf} 
\end{center} 
\subsubsection*{A1-03 -- Définir les domaines d’application et les critères technico-économiques et environnementaux.}  
\begin{center} 
\includegraphics{A1-03.pdf} 
\end{center} 
\subsubsection*{A1-04 -- Qualifier et quantifier les exigences.}  
\begin{center} 
\includegraphics{A1-04.pdf} 
\end{center} 
\subsubsection*{A1-05 -- Évaluer l’impact environnemental et sociétal.}  
\begin{center} 
\includegraphics{A1-05.pdf} 
\end{center} 
\subsection{A2 -- Définir les frontières de l'analyse}  
\subsubsection*{A2-01 -- Isoler un système et justifier l’isolement.}  
\begin{center} 
\includegraphics{A2-01.pdf} 
\end{center} 
\subsubsection*{A2-02 -- Définir les éléments influents du milieu extérieur. }  
\begin{center} 
\includegraphics{A2-02.pdf} 
\end{center} 
\subsubsection*{A2-03 -- Identifier la nature des flux échangés traversant la frontière d’étude.}  
\begin{center} 
\includegraphics{A2-03.pdf} 
\end{center} 
\subsection{A3 -- Analyser l'organisation fonctionnelle et structurelle}  
\subsubsection*{A3-01 -- Associer les fonctions aux constituants.}  
\begin{center} 
\includegraphics{A3-01.pdf} 
\end{center} 
\subsubsection*{A3-02 -- Justifier le choix des constituants dédiés aux fonctions d’un système.}  
\begin{center} 
\includegraphics{A3-02.pdf} 
\end{center} 
\subsubsection*{A3-03 -- Identifier et décrire les chaines fonctionnelles du système.}  
\begin{center} 
\includegraphics{A3-03.pdf} 
\end{center} 
\subsubsection*{A3-04 -- Identifier et décrire les liens entre les chaines fonctionnelles.}  
\subsubsection*{A3-05 -- Caractériser un constituant de la chaine de puissance.}  
\subsubsection*{A3-06 -- Caractériser un constituant de la chaine d’information.}  
\subsubsection*{A3-07 -- Analyser un algorithme. }  
\subsubsection*{A3-08 -- Analyser les principes d'intelligence artificielle. }  
\subsubsection*{A3-09 -- Interpréter tout ou partie de l’évolution temporelle d’un système séquentiel.}  
\subsubsection*{A3-10 -- Identifier la structure d'un système asservi.}  
\subsection{A4 -- Analyser les performances et les écarts}  
\subsubsection*{A4-01 -- Extraire un indicateur de performance pertinent à partir du cahier des charges ou de résultats issus de l'expérimentation ou de la simulation.}  
\subsubsection*{A4-02 -- Caractériser les écarts entre les performances.}  
\subsubsection*{A4-03 -- Interpréter et vérifier la cohérence des résultats obtenus expérimentalement, analytiquement ou numériquement. }  
\subsubsection*{A4-04 -- Rechercher et proposer des causes aux écarts constatés.}  
\section{B -- Modéliser}  
\subsection{B1 -- Choisir les grandeurs physiques et les caractériser}  
\subsubsection*{B1-01 -- Identifier les performances à prévoir ou à évaluer.}  
\subsubsection*{B1-02 -- Identifier les grandeurs d'entrée et de sortie d’un modèle.}  
\subsubsection*{B1-03 -- Identifier les paramètres d’un modèle.}  
\subsubsection*{B1-04 -- Identifier et justifier les hypothèses nécessaires à la modélisation.}  
\subsection{B2 -- Proposer un modèle de connaissance et de comportement}  
\subsubsection*{B2-01 -- Choisir un modèle adapté aux performances à prévoir ou à évaluer.}  
\subsubsection*{B2-02 -- Compléter un modèle multiphysique.}  
\subsubsection*{B2-03 -- Associer un modèle aux composants des chaines fonctionnelles.}  
\subsubsection*{B2-04 -- Établir un modèle de connaissance par des fonctions de transfert.}  
\subsubsection*{B2-05 -- Modéliser le signal d'entrée.}  
\subsubsection*{B2-06 -- Établir un modèle de comportement à partir d'une réponse temporelle ou fréquentielle. }  
\subsubsection*{B2-07 -- Modéliser un système par schéma-blocs. }  
\subsubsection*{B2-08 -- Simplifier un modèle.}  
\subsubsection*{B2-09 -- Modéliser un correcteur numérique. }  
\subsubsection*{B2-10 -- Déterminer les caractéristiques d'un solide ou d'un ensemble de solides indéformables.}  
\subsubsection*{B2-11 -- Proposer une modélisation des liaisons avec leurs caractéristiques géométriques.}  
\subsubsection*{B2-12 -- Proposer un modèle cinématique à partir d'un système réel ou d'une maquette numérique.}  
\subsubsection*{B2-13 -- Modéliser la cinématique d'un ensemble de solides.}  
\subsubsection*{B2-14 -- Modéliser une action mécanique.}  
\subsubsection*{B2-15 -- Simplifier un modèle de mécanisme.}  
\subsubsection*{B2-16 -- Modifier un modèle pour le rendre isostatique.}  
\subsubsection*{B2-17 -- Décrire le comportement d'un système séquentiel.}  
\subsection{B3 -- Valider un modèle}  
\subsubsection*{B3-01 -- Vérifier la cohérence du modèle choisi en confrontant les résultats analytiques et/ou numériques aux résultats expérimentaux.}  
\subsubsection*{B3-02 -- Préciser les limites de validité d'un modèle.}  
\subsubsection*{B3-03 -- Modifier les paramètres et enrichir le modèle pour minimiser l’écart entre les résultats analytiques et/ou numériques et les résultats expérimentaux.}  
\section{C -- Résoudre}  
\subsection{C1 -- Proposer une démarche de résolution}  
\subsubsection*{C1-01 -- Proposer une démarche permettant d'évaluer les performances des systèmes asservis.}  
\subsubsection*{C1-02 -- Proposer une démarche de réglage d'un correcteur.}  
\subsubsection*{C1-03 -- Choisir une démarche de résolution d’un problème d'ingénierie numérique ou d'intelligence artificielle. }  
\subsubsection*{C1-04 -- Proposer une démarche permettant d'obtenir une loi entrée-sortie géométrique. }  
\subsubsection*{C1-05 -- Proposer une démarche permettant la détermination d’une action mécanique inconnue ou d'une loi de mouvement.}  
\subsection{C2 -- Mettre en œuvre une démarche de résolution analytique}  
\subsubsection*{C2-01 -- Déterminer la réponse temporelle.}  
\subsubsection*{C2-02 -- Déterminer la réponse fréquentielle. }  
\subsubsection*{C2-03 -- Déterminer les performances d'un système asservi.}  
\subsubsection*{C2-04 -- Mettre en œuvre une démarche de réglage d’un correcteur.}  
\subsubsection*{C2-05 -- Caractériser le mouvement d’un repère par rapport à un autre repère.}  
\subsubsection*{C2-06 -- Déterminer les relations entre les grandeurs géométriques ou cinématiques. }  
\subsubsection*{C2-07 -- Déterminer les actions mécaniques en statique.}  
\subsubsection*{C2-08 -- Déterminer les actions mécaniques en dynamique dans le cas où le mouvement est imposé.}  
\subsubsection*{C2-09 -- Déterminer la loi de mouvement dans le cas où les efforts extérieurs sont connus.}  
\subsection{C3 -- Mettre en œuvre une démarche de résolution numérique}  
\subsubsection*{C3-01 -- Mener une simulation numérique. }  
\subsubsection*{C3-02 -- Résoudre numériquement une équation ou un système d'équations. }  
\subsubsection*{C3-03 -- Résoudre un problème en utilisant une solution d'intelligence artificielle. }  
\section{D -- Expérimenter}  
\subsection{D1 -- Mettre en œuvre un système}  
\subsubsection*{D1-01 -- Mettre en œuvre un système en suivant un protocole.}  
\subsubsection*{D1-02 -- Repérer les constituants réalisant les principales fonctions des chaines fonctionnelles.}  
\subsubsection*{D1-03 -- Identifier les grandeurs physiques d’effort et de flux.}  
\subsection{D2 -- Proposer et justifier un protocole expérimental}  
\subsubsection*{D2-01 -- Choisir le protocole en fonction de l'objectif visé.}  
\subsubsection*{D2-02 -- Choisir les configurations matérielles et logicielles du système en fonction de l'objectif visé par l'expérimentation.}  
\subsubsection*{D2-03 -- Choisir les réglages du système en fonction de l'objectif visé par l'expérimentation.}  
\subsubsection*{D2-04 -- Choisir la grandeur physique à mesurer ou justifier son choix.}  
\subsubsection*{D2-05 -- Choisir les entrées à imposer et les sorties pour identifier un modèle de comportement.}  
\subsubsection*{D2-06 -- Justifier le choix d’un capteur ou d’un appareil de mesure vis-à-vis de la grandeur physique à mesurer.}  
\subsection{D3 -- Mettre en œuvre un protocole expérimental}  
\subsubsection*{D3-01 -- Régler les paramètres de fonctionnement d'un système.}  
\subsubsection*{D3-02 -- Mettre en œuvre un appareil de mesure adapté à la caractéristique de la grandeur à mesurer.}  
\subsubsection*{D3-03 -- Effectuer des traitements à partir de données. }  
\subsubsection*{D3-04 -- Identifier les erreurs de mesure.}  
\subsubsection*{D3-05 -- Identifier les erreurs de méthode.}  
\section{E -- Communiquer}  
\subsection{E1 -- Rechercher et traiter des informations}  
\subsubsection*{E1-01 -- Rechercher des informations.}  
\subsubsection*{E1-02 -- Distinguer les différents types de documents et de données en fonction de leurs usages.}  
\subsubsection*{E1-03 -- Vérifier la pertinence des informations (obtention, véracité, fiabilité et précision de l'information).}  
\subsubsection*{E1-04 -- Extraire les informations utiles d’un dossier technique.}  
\subsubsection*{E1-05 -- Lire et décoder un document technique.}  
\subsubsection*{E1-06 -- Trier les informations selon des critères.}  
\subsubsection*{E1-07 -- Effectuer une synthèse des informations disponibles dans un dossier technique.}  
\subsection{E2 -- Produire et échanger de l'information}  
\subsubsection*{E2-01 -- Choisir un outil de communication adapté à l’interlocuteur.}  
\subsubsection*{E2-02 -- Faire preuve d’écoute et confronter des points de vue.}  
\subsubsection*{E2-03 -- Présenter les étapes de son travail.}  
\subsubsection*{E2-04 -- Présenter de manière argumentée une synthèse des résultats.}  
\subsubsection*{E2-05 -- Produire des documents techniques adaptés à l'objectif de la communication. }  
\subsubsection*{E2-06 -- Utiliser un vocabulaire technique, des symboles et des unités adéquats.}  
\section{F -- Concevoir}  
\subsection{F1 -- Concevoir l'architecture d'un système innovant}  
\subsubsection*{F1-01 -- Proposer une architecture fonctionnelle et organique.}  
\subsection{F2 -- Proposer et choisir des solutions techniques}  
\subsubsection*{F2-01 -- Modifier la commande pour faire évoluer le comportement du système. }  


\end{document}
