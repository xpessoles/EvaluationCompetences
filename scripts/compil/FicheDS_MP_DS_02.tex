\documentclass[10pt,fleqn]{article} % Default font size and left-justified equations
\usepackage[%
    pdftitle={Modélisation systèmes multiphysiques : Modélisation par fonction de transfert et schéma-blocs},
    pdfauthor={Xavier Pessoles}]{hyperref}
\input{style/new_style}
\input{style/macros_SII}
\def\classe{\textsf{PSI$\star$}}
\def\xxnumpartie{}
\def\xxpartie{}%Modéliser et corriger le comportement linéaire et non linéaire des systèmes multiphysiques}
\def\xxnumchapitre{Devoir surveillé de SII 3\vspace{.2cm}}
\def\xxchapitre{\hspace{.12cm}  }
\def\discipline{Sciences \\Industrielles de \\ l'Ingénieur}
\def\xxtete{Sciences Industrielles de l'Ingénieur}
\def\xxactivite{\textsf{DS 3}}
\def\xxonglet{\textsf{DS 3}}
\def\xxauteur{\textsl{Xavier Pessoles}}

\def\xxtitreexo{\noindent Table Azalée}
\def\xxsourceexo{\hspace{.2cm} Banque PT -- SIA 2019}


\def\xxfigures{\includegraphics[width=.6\linewidth]{fig_00}}
\def\xxpied{%
\textsf{DS 3}%dans le but de déterminer les contraintes géométriques dans les mécanismes\\% afin de valider leurs performances.\\
%Révisions 1 -- 2 -- 3 -- \xxactivite%
}

\def\xxposongletx{2}
\def\xxposonglettext{1.45}
\def\xxposonglety{20}
%\def\xxonglet{Part. 1 -- Ch. 3}


\usepackage{multicol}
\usepackage{siunitx}

\fichetrue
%\fichefalse

\proftrue
\proffalse

\tdtrue
\tdfalse

\courstrue
\coursfalse

\setcounter{secnumdepth}{5}
%---------------------------------------------------------------------------


\begin{document}
%\chapterimage{png/Fond_Cin}
\input{style/new_pagegarde}
%\vspace{4.5cm}
\pagestyle{fancy}
\thispagestyle{plain}


\def\columnseprulecolor{\color{ocre}}
\setlength{\columnseprule}{0.4pt} 


\begin{tikzpicture}[remember picture,overlay]
\draw[anchor=west] (-2cm,-4cm) node {\huge\sffamily\bfseries\color{black} %
\begin{minipage}{5cm}
\begin{center}
\LARGE\sffamily\color{ocre}\textbf{\textsc{\xxactivite}}

\begin{center}
\xxfigures
\end{center}

\end{center}
\end{minipage} \hfill
\begin{minipage}[c]{12cm}
\begin{titrechapitre}
\renewcommand{\baselinestretch}{1.1} 
\large\sffamily\textbf{\textsc{\xxtitreexo}}

\small\sffamily{\textbf{\textit{\color{black!70}\xxsourceexo}}}
\vspace{.5cm}

\renewcommand{\baselinestretch}{1} 
\normalsize\normalfont
%\xxcompetences
\end{titrechapitre}
\end{minipage}};
\end{tikzpicture}

\vspace{6cm}


\textbf{Remarques générales -- Rapport du jury}
\begin{itemize}
\item Q1 - Une question classique d’étude d’une loi de vitesse en trapèze. Il faut veiller à répondre correctement
à la question, en donnant les expressions littérales demandées. Le résultat seul ne suffit pas, et les
applications numériques doivent être correctement effectuées.
\item Q2 - Une question destinée à préparer la Q3. Beaucoup d’erreurs ont été commises suite à un mauvais
Bilan des Actions Mécaniques Extérieures s’exerçant sur la roue. Identifier un cas simple permet de
s’affranchir de calculs inutilement compliqués.
\item Q3 - Application du Théorème du Moment Dynamique à un solide unique. Cette question a été bien
réussie quand la Q2 l’a été également. Trop de réponses sont données sans se soucier de la forme
demandée dans la question. C’est dommage de perdre des points pour cela alors que la méthode est
bonne.
\item Q4 - Cette question interrogeait le candidat sur sa capacité à choisir la méthode de résolution du
problème posé. Choisir la bonne équation à écrire permet d’optimiser la résolution du problème.
Cependant, même lorsque la méthode choisie était la bonne, les calculs menés pour déterminer le
moment dynamique n’ont que trop rarement abouti.
\item Q5 - Comme pour Q4, cette question évaluait plutôt la capacité du candidat à définir une démarche
de résolution. Aucun calcul n’était demandé, mais on peut noter qu’une fois encore, le Bilan des
Actions Mécaniques Extérieures a été trop souvent négligé. Le jury souhaite mettre l’accent sur le
fait qu’annoncer un Bilan des Actions Mécaniques Extérieures et un théorème sans avoir au préalable
indiqué l’isolement n’a pas de sens ! Le jury a donc sanctionné les candidats qui n’auraient pas indiqué
l’isolement.
\item Q6 - Cette question permettait de relier les calculs menés de Q2 à Q5 à la problématique de la partie
1. En utilisant le modèle de Coulomb, visiblement mal maitrisé par les candidats, il suffisait de relier le
facteur de frottement aux actions mécaniques déterminées précédemment. Attention à ne pas oublier
les valeurs absolues dans le calcul lorsqu’on établit le facteur de frottement nécessaire.
\item Q7 - Les valeurs étant données, le candidat devait simplement prendre en considération la simplification
du modèle (passage de 2 à 4 roues) et le coefficient de sécurité pour définir le facteur de frottement
minimal à prendre en compte pour choisir le même couple de matériaux en jeu. Une fois encore, trop
de réponses ont été incomplètes ou incorrectes, du fait d’une mauvaise compréhension des phénomènes
et d’une mauvaise lecture de l’énoncé.
\item Q8 - Cette question avec la suivante (Q9) étaient liées et indépendantes. Elles ont été très souvent
traitées. L’objectif était de vérifier un critère de temps de levage du chariot qui ne doit pas rester
en appui sur les rails pour ne pas déformer le bandage de roulement. Rares sont les candidats à
avoir répondu complètement juste à cette question qui ne demandait que l’identification et la bonne
modélisation cinématique de 3 transmetteurs usuels afin de déterminer une vitesse de sortie d’une
chaine de puissance à partir des données techniques et du paramètre cinématique d’entrée fournis.
\item Q9 - Le jury rappelle à l’occasion de cette question que la valeur ou le nom du critère à vérifier doivent
être explicitement indiqués pour un retour au cahier des charges correct.
\item Q10 - Cette première question d’asservissement demandait aux candidats d’analyser un schéma-blocs
et d’en déduire le modèle de 3 gains. En particulier, les candidats étaient invités à bien faire apparaitre
les unités et devaient maitriser une condition « de bon fonctionnement ».
\item Q11 - Cette question, destinée à préparer la question Q12, demandait de calculer 2 fonctions de
transfert d’un asservissement, l’une vis-à-vis de la seule consigne et l’autre vis-à-vis de la seule
perturbation. C’est dommage que certains candidats n’aient pas pris le temps de bien lire les définitions
des grandeurs et se soient trompés de signe. Une question pratiquement toujours traitée avec beaucoup
de réussite.
\item Q12 - Une question bien traitée en général quand la question Q11 l’avait été correctement.
\item Q13 - Une question de cours : le cahier des charges impose un écart statique nul et un correcteur
proportionnel n’est pas capable d’annuler un écart statique initialement non nul. Le jury a pénalisé
les candidats qui ont multiplié les critères qui ne seraient pas (selon eux) vérifiés. En particulier, un
correcteur proportionnel peut faire varier les marges de stabilité et la rapidité d’un système du second
ordre. Cette question justifiait pourquoi le sujet changeait de correcteur par la suite.
\item Q14 - Une question classique qui évaluait la capacité des candidats à déterminer une valeur limite
du gain du correcteur proportionnel-Intégral à partir d’un diagramme de Bode de la FTBO afin de
satisfaire les marges de stabilité. La méthode graphique était la plus efficace possible. Le jury trouve
dommage que cette compétence classique n’ait pas été suffisamment bien maîtrisée.
\item Q15 - Cette question très bien traitée, destinée à préparer la Q16, demandait d’identifier les paramètres
caractéristiques d’un système du second ordre.
\item Q16 - Une question bien traitée en général quand la question Q15 l’avait été correctement. Attention
toutefois, le jury rappelle que la valeur optimale (approchée) du coefficient d’amortissement pour que
la réponse indicielle d’un système fondamental du second ordre soit le plus rapide est 0,69 (voire 0, 7)
mais pas $\sqrt{2}/2$ ! Bien évidemment le jury a naturellement accepté la valeur de $\sqrt{2}/2$ pour la SEULE
application numérique.
\item Q17 - Une question bilan de la partie asservissement. Une réponse indicielle du système corrigé avec
la bonne valeur du gain du proportionnel-Intégral était fournie. Le candidat était amené à déterminer
le temps de réponse à 5\% et de conclure. La conclusion n’était admise que si le candidat rappelait bien
la valeur ou le nom de chacun des 3 critères à vérifier dans le cahier des charges.
\item Q18 - Une question qui demandait de réaliser une fermeture géométrique, qui nécessitait de savoir
projeter un vecteur dans une base et de connaitre deux démarches classiques de l’ingénieur : l’élimination
d’une inconnue et une linéarisation autour d’un point de fonctionnement.
\item Q19 - Une question liée à la précédente qui demandait d’inverser et de dériver le résultat précédent.
Le jury a valorisé la démarche tant que le résultat littéral était homogène et cohérent avec la question
précédente.
\item Q20 - Une question de cinématique de base pour la détermination de la vitesse ; il ne fallait pas
se tromper de signe, ni de direction. Le candidat devait connaitre la relation simple dF = p.dS et
savoir que la force élémentaire est perpendiculaire à la surface d’appui en l’absence de frottement pour
déterminer le vecteur force élémentaire. Le jury se désole de voir que certains candidats ne prennent
pas assez de recul pour se rendre compte que la puissance galiléenne élémentaire recherchée (action de
l’eau sur le batteur) est négative (perdue).
\item Q21 - Le jury a mis l’accent sur la démarche : l’expression de la puissance galiléenne globale est
l’intégrale sur la surface de la puissance galiléenne élémentaire. L’expression de la surface élémentaire
était nécessaire et les bornes d’intégrations étaient à indiquer. Le jury se désole que certains candidats
n’arrivent pas à réaliser une intégration simple.
\item Q22 - Le candidat qui avait bien suivi le raisonnement de cette partie arrivait ici en ayant constaté
précédemment que l’expression de la puissance galiléenne maximale dissipée par l’action de l’eau sur le
batteur dépendait de 2 paramètres : l’amplitude du batteur et la pulsation du batteur. Cette question
assez indépendante a été bien traitée. L’objectif était de déterminer le produit « hauteur de houle x
fréquence de houle » le plus grand.
\item Q23 - Dans cette question, on faisait l’hypothèse que les 2 grandeurs étudiées étaient reliées par
une équation différentielle linéaire à coefficients constants. De fait, le candidat devait procéder en 2
étapes : d’abord, à une entrée sinusoïdale correspond une sortie sinusoïdale de même pulsation (sous
les hypothèses – classiques – mentionnées) et déphasée. Ensuite la courbe de gain en décibel de la
fonction de transfert reliant les 2 grandeurs permettait aux candidats de déterminer l’amplitude du
batteur connaissant la pulsation d’excitation et la hauteur de houle désirées.
\item Q24 - Première question d’un dernier triptyque permettant de s’assurer que le groupe hydraulique
proposé était suffisamment dimensionné. Dans cette question, une méthode énergétique était demandée
pour déterminer l’expression et la valeur de l’effort du vérin nécessaire.
\item Q25 - Une question qui demandait au candidat de savoir interpréter un plan d’ensemble d’un vérin
double effet à tige traversante et d’en proposer l’expression d’une surface utile. L’objectif était de
déterminer la valeur maximale de l’effort du vérin en considérant que le vérin avait un rendement
non unitaire. Un retour avec une conclusion sur ce résultat avec celui de la question précédente était
attendu.
\item Q26 - Une dernière question qui permettait de justifier que la puissance électrique du groupe hydraulique
permettant de générer une houle la plus énergivore était suffisante. La prise en compte d’une puissance
galiléenne maximale de dissipation (fournie) par l’action de l’eau sur le batteur, de rendements de
divers composants du mécanisme et d’un coefficient de sécurité permettait d’aboutir à un résultat et
de conclure.
\end{itemize}

Les questions Q18 à Q20 ainsi que la question Q22 ont été assez bien traitées et abordées par de
nombreux candidats.
Les autres questions de cette partie ont été par contre assez peu traitées. A noter cependant que les
candidats qui auront traité les questions Q24 et Q26 s’en s’ont plutôt bien sorti.



\begin{minipage}[c]{.45\linewidth} 
\Large \textbf{\textsf{NOM19 Prenom19}} 
 
 \normalsize Note harmonisée 6.11/20 
 
Rang 6
 
Moyenne classe harmonisée 7.13/20 
 
Moyenne question traitées 6.82/20 
 
Rang question traitées 3 
 
Commentaires : 
Comment20 
\end{minipage}\hfill 
\begin{minipage}[c]{.45\linewidth}  
\begin{center}
\includegraphics[width=.8\linewidth]{../histo.pdf} 
\end{center}
\end{minipage}
\footnotesize 
\begin{center} 
\begin{tabular}{|c|c|m{1cm}|c||c|c|m{1cm}|c||c|c|m{1cm}|c||c|c|m{1cm}|c|} 
\hline \textbf{Qu} & \textbf{Coef} & \textbf{Comp} & \textbf{/5} & \textbf{Qu} & \textbf{Coef} & \textbf{Comp} & \textbf{/5} & \textbf{Qu} & \textbf{Coef} & \textbf{Comp} & \textbf{/5} & \textbf{Qu} & \textbf{Coef} & \textbf{Comp} & \textbf{/5} \\ 
\hline 
\hline 
Q1 & 5 & A1-01 & NT & Q1 & 5 & A1-02 & 2 & Q2 & 5 & A1-03 & 5 & Q2 & 5 & A1-04 & 0 \\ \hline 
 
Q3 & 5 & A1-05 & 1 & Q4 & 5 & A2-01 & 4 & Q5 & 5 & A2-02 & NT & Q6 & 5 & A2-03 & 2 \\ \hline 
 
Q7 & 5 & A3-01 & 5 & Q8 & 5 & A3-02 & 0 & Q9 & 10 & A3-03 & 1 &  &  &  &  \\ \hline 
 
\end{tabular} 
\end{center} 
\normalsize 
 


%\section{A -- Analyser}  
\subsection{A1 -- Analyser le besoin et les exigences}  
\subsubsection*{A1-01 -- Décrire le besoin et les exigences.}  
\begin{center} 
\includegraphics{A1-01.pdf} 
\end{center} 
\subsubsection*{A1-02 -- Traduire un besoin fonctionnel en exigences.}  
\begin{center} 
\includegraphics{A1-02.pdf} 
\end{center} 
\subsubsection*{A1-03 -- Définir les domaines d’application et les critères technico-économiques et environnementaux.}  
\begin{center} 
\includegraphics{A1-03.pdf} 
\end{center} 
\subsubsection*{A1-04 -- Qualifier et quantifier les exigences.}  
\begin{center} 
\includegraphics{A1-04.pdf} 
\end{center} 
\subsubsection*{A1-05 -- Évaluer l’impact environnemental et sociétal.}  
\begin{center} 
\includegraphics{A1-05.pdf} 
\end{center} 
\subsection{A2 -- Définir les frontières de l'analyse}  
\subsubsection*{A2-01 -- Isoler un système et justifier l’isolement.}  
\begin{center} 
\includegraphics{A2-01.pdf} 
\end{center} 
\subsubsection*{A2-02 -- Définir les éléments influents du milieu extérieur. }  
\begin{center} 
\includegraphics{A2-02.pdf} 
\end{center} 
\subsubsection*{A2-03 -- Identifier la nature des flux échangés traversant la frontière d’étude.}  
\begin{center} 
\includegraphics{A2-03.pdf} 
\end{center} 
\subsection{A3 -- Analyser l'organisation fonctionnelle et structurelle}  
\subsubsection*{A3-01 -- Associer les fonctions aux constituants.}  
\begin{center} 
\includegraphics{A3-01.pdf} 
\end{center} 
\subsubsection*{A3-02 -- Justifier le choix des constituants dédiés aux fonctions d’un système.}  
\begin{center} 
\includegraphics{A3-02.pdf} 
\end{center} 
\subsubsection*{A3-03 -- Identifier et décrire les chaines fonctionnelles du système.}  
\begin{center} 
\includegraphics{A3-03.pdf} 
\end{center} 
\subsubsection*{A3-04 -- Identifier et décrire les liens entre les chaines fonctionnelles.}  
\subsubsection*{A3-05 -- Caractériser un constituant de la chaine de puissance.}  
\subsubsection*{A3-06 -- Caractériser un constituant de la chaine d’information.}  
\subsubsection*{A3-07 -- Analyser un algorithme. }  
\subsubsection*{A3-08 -- Analyser les principes d'intelligence artificielle. }  
\subsubsection*{A3-09 -- Interpréter tout ou partie de l’évolution temporelle d’un système séquentiel.}  
\subsubsection*{A3-10 -- Identifier la structure d'un système asservi.}  
\subsection{A4 -- Analyser les performances et les écarts}  
\subsubsection*{A4-01 -- Extraire un indicateur de performance pertinent à partir du cahier des charges ou de résultats issus de l'expérimentation ou de la simulation.}  
\subsubsection*{A4-02 -- Caractériser les écarts entre les performances.}  
\subsubsection*{A4-03 -- Interpréter et vérifier la cohérence des résultats obtenus expérimentalement, analytiquement ou numériquement. }  
\subsubsection*{A4-04 -- Rechercher et proposer des causes aux écarts constatés.}  
\section{B -- Modéliser}  
\subsection{B1 -- Choisir les grandeurs physiques et les caractériser}  
\subsubsection*{B1-01 -- Identifier les performances à prévoir ou à évaluer.}  
\subsubsection*{B1-02 -- Identifier les grandeurs d'entrée et de sortie d’un modèle.}  
\subsubsection*{B1-03 -- Identifier les paramètres d’un modèle.}  
\subsubsection*{B1-04 -- Identifier et justifier les hypothèses nécessaires à la modélisation.}  
\subsection{B2 -- Proposer un modèle de connaissance et de comportement}  
\subsubsection*{B2-01 -- Choisir un modèle adapté aux performances à prévoir ou à évaluer.}  
\subsubsection*{B2-02 -- Compléter un modèle multiphysique.}  
\subsubsection*{B2-03 -- Associer un modèle aux composants des chaines fonctionnelles.}  
\subsubsection*{B2-04 -- Établir un modèle de connaissance par des fonctions de transfert.}  
\subsubsection*{B2-05 -- Modéliser le signal d'entrée.}  
\subsubsection*{B2-06 -- Établir un modèle de comportement à partir d'une réponse temporelle ou fréquentielle. }  
\subsubsection*{B2-07 -- Modéliser un système par schéma-blocs. }  
\subsubsection*{B2-08 -- Simplifier un modèle.}  
\subsubsection*{B2-09 -- Modéliser un correcteur numérique. }  
\subsubsection*{B2-10 -- Déterminer les caractéristiques d'un solide ou d'un ensemble de solides indéformables.}  
\subsubsection*{B2-11 -- Proposer une modélisation des liaisons avec leurs caractéristiques géométriques.}  
\subsubsection*{B2-12 -- Proposer un modèle cinématique à partir d'un système réel ou d'une maquette numérique.}  
\subsubsection*{B2-13 -- Modéliser la cinématique d'un ensemble de solides.}  
\subsubsection*{B2-14 -- Modéliser une action mécanique.}  
\subsubsection*{B2-15 -- Simplifier un modèle de mécanisme.}  
\subsubsection*{B2-16 -- Modifier un modèle pour le rendre isostatique.}  
\subsubsection*{B2-17 -- Décrire le comportement d'un système séquentiel.}  
\subsection{B3 -- Valider un modèle}  
\subsubsection*{B3-01 -- Vérifier la cohérence du modèle choisi en confrontant les résultats analytiques et/ou numériques aux résultats expérimentaux.}  
\subsubsection*{B3-02 -- Préciser les limites de validité d'un modèle.}  
\subsubsection*{B3-03 -- Modifier les paramètres et enrichir le modèle pour minimiser l’écart entre les résultats analytiques et/ou numériques et les résultats expérimentaux.}  
\section{C -- Résoudre}  
\subsection{C1 -- Proposer une démarche de résolution}  
\subsubsection*{C1-01 -- Proposer une démarche permettant d'évaluer les performances des systèmes asservis.}  
\subsubsection*{C1-02 -- Proposer une démarche de réglage d'un correcteur.}  
\subsubsection*{C1-03 -- Choisir une démarche de résolution d’un problème d'ingénierie numérique ou d'intelligence artificielle. }  
\subsubsection*{C1-04 -- Proposer une démarche permettant d'obtenir une loi entrée-sortie géométrique. }  
\subsubsection*{C1-05 -- Proposer une démarche permettant la détermination d’une action mécanique inconnue ou d'une loi de mouvement.}  
\subsection{C2 -- Mettre en œuvre une démarche de résolution analytique}  
\subsubsection*{C2-01 -- Déterminer la réponse temporelle.}  
\subsubsection*{C2-02 -- Déterminer la réponse fréquentielle. }  
\subsubsection*{C2-03 -- Déterminer les performances d'un système asservi.}  
\subsubsection*{C2-04 -- Mettre en œuvre une démarche de réglage d’un correcteur.}  
\subsubsection*{C2-05 -- Caractériser le mouvement d’un repère par rapport à un autre repère.}  
\subsubsection*{C2-06 -- Déterminer les relations entre les grandeurs géométriques ou cinématiques. }  
\subsubsection*{C2-07 -- Déterminer les actions mécaniques en statique.}  
\subsubsection*{C2-08 -- Déterminer les actions mécaniques en dynamique dans le cas où le mouvement est imposé.}  
\subsubsection*{C2-09 -- Déterminer la loi de mouvement dans le cas où les efforts extérieurs sont connus.}  
\subsection{C3 -- Mettre en œuvre une démarche de résolution numérique}  
\subsubsection*{C3-01 -- Mener une simulation numérique. }  
\subsubsection*{C3-02 -- Résoudre numériquement une équation ou un système d'équations. }  
\subsubsection*{C3-03 -- Résoudre un problème en utilisant une solution d'intelligence artificielle. }  
\section{D -- Expérimenter}  
\subsection{D1 -- Mettre en œuvre un système}  
\subsubsection*{D1-01 -- Mettre en œuvre un système en suivant un protocole.}  
\subsubsection*{D1-02 -- Repérer les constituants réalisant les principales fonctions des chaines fonctionnelles.}  
\subsubsection*{D1-03 -- Identifier les grandeurs physiques d’effort et de flux.}  
\subsection{D2 -- Proposer et justifier un protocole expérimental}  
\subsubsection*{D2-01 -- Choisir le protocole en fonction de l'objectif visé.}  
\subsubsection*{D2-02 -- Choisir les configurations matérielles et logicielles du système en fonction de l'objectif visé par l'expérimentation.}  
\subsubsection*{D2-03 -- Choisir les réglages du système en fonction de l'objectif visé par l'expérimentation.}  
\subsubsection*{D2-04 -- Choisir la grandeur physique à mesurer ou justifier son choix.}  
\subsubsection*{D2-05 -- Choisir les entrées à imposer et les sorties pour identifier un modèle de comportement.}  
\subsubsection*{D2-06 -- Justifier le choix d’un capteur ou d’un appareil de mesure vis-à-vis de la grandeur physique à mesurer.}  
\subsection{D3 -- Mettre en œuvre un protocole expérimental}  
\subsubsection*{D3-01 -- Régler les paramètres de fonctionnement d'un système.}  
\subsubsection*{D3-02 -- Mettre en œuvre un appareil de mesure adapté à la caractéristique de la grandeur à mesurer.}  
\subsubsection*{D3-03 -- Effectuer des traitements à partir de données. }  
\subsubsection*{D3-04 -- Identifier les erreurs de mesure.}  
\subsubsection*{D3-05 -- Identifier les erreurs de méthode.}  
\section{E -- Communiquer}  
\subsection{E1 -- Rechercher et traiter des informations}  
\subsubsection*{E1-01 -- Rechercher des informations.}  
\subsubsection*{E1-02 -- Distinguer les différents types de documents et de données en fonction de leurs usages.}  
\subsubsection*{E1-03 -- Vérifier la pertinence des informations (obtention, véracité, fiabilité et précision de l'information).}  
\subsubsection*{E1-04 -- Extraire les informations utiles d’un dossier technique.}  
\subsubsection*{E1-05 -- Lire et décoder un document technique.}  
\subsubsection*{E1-06 -- Trier les informations selon des critères.}  
\subsubsection*{E1-07 -- Effectuer une synthèse des informations disponibles dans un dossier technique.}  
\subsection{E2 -- Produire et échanger de l'information}  
\subsubsection*{E2-01 -- Choisir un outil de communication adapté à l’interlocuteur.}  
\subsubsection*{E2-02 -- Faire preuve d’écoute et confronter des points de vue.}  
\subsubsection*{E2-03 -- Présenter les étapes de son travail.}  
\subsubsection*{E2-04 -- Présenter de manière argumentée une synthèse des résultats.}  
\subsubsection*{E2-05 -- Produire des documents techniques adaptés à l'objectif de la communication. }  
\subsubsection*{E2-06 -- Utiliser un vocabulaire technique, des symboles et des unités adéquats.}  
\section{F -- Concevoir}  
\subsection{F1 -- Concevoir l'architecture d'un système innovant}  
\subsubsection*{F1-01 -- Proposer une architecture fonctionnelle et organique.}  
\subsection{F2 -- Proposer et choisir des solutions techniques}  
\subsubsection*{F2-01 -- Modifier la commande pour faire évoluer le comportement du système. }  


\end{document}
