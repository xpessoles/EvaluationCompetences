\begin{minipage}[c]{.45\linewidth} 
\Large \textbf{\textsf{XXXXX XXXxxxxxx}} 
 
 \normalsize Note brute 6.65/20 
 
 \normalsize Note harmonisée 9.31/20 
 
Rang 27
 
Moyenne classe brute 6.75/20 
 
Moyenne question traitées 8.31/20 
 
Rang question traitées 32 
 
Commentaires : 
Un ensemble un peu inégal. Sur ce genre de sujet, il ne faut traiter de manière exhaustive les questions abordées.(Pour le cahier des charges, il faut par exemple évaluer chacun des critères).  
\end{minipage}\hfill 
\begin{minipage}[c]{.45\linewidth}  
\begin{center}
\includegraphics[width=.8\linewidth]{../histo.pdf} 
\end{center}
\end{minipage}
\footnotesize 
\begin{center} 
\begin{tabular}{|c|c|m{1cm}|c||c|c|m{1cm}|c||c|c|m{1cm}|c||c|c|m{1cm}|c|} 
\hline \textbf{Qu} & \textbf{Coef} & \textbf{Comp} & \textbf{/5} & \textbf{Qu} & \textbf{Coef} & \textbf{Comp} & \textbf{/5} & \textbf{Qu} & \textbf{Coef} & \textbf{Comp} & \textbf{/5} & \textbf{Qu} & \textbf{Coef} & \textbf{Comp} & \textbf{/5} \\ 
\hline 
\hline 
Q1 & 1 & GEO-01 & 3.5 & Q1 & 2 & GEO-02 & NT & Q2 & 1 & GEO-03 & 0 & Q3 & 2 & GEO-04 & 2 \\ \hline 
 
Q3 & 1 & CIN-01 & 0 & Q4 & 2 & CIN-02 & 0 & Q5 & 1 & CIN-03 & 3 & Q6 & 2 & CIN-04 & 2 \\ \hline 
 
Q7 & 1 & CIN-05 & 5 & Q8 & 2 & STAT-01 & 4 & Q9 & 1 & STAT-02 & NT & Q10 & 2 & STAT-03 & 0 \\ \hline 
 
Q11 & 1 & STAT-04 & 0 & Q12 & 2 & STAT-05 & 5 & Q13 & 1 & CHS-01 & 5 & Q14 & 2 & CHS-02 & 5 \\ \hline 
 
Q15 & 1 & CHS-03 & 0 & Q16 & 2 & CHS-04 & 0 & Q17 & 1 & CHS-05 & NT & Q18 & 2 & DYN-01 & NT \\ \hline 
 
Q19 & 1 & DYN-02 & 1 & Q20 & 2 & DYN-03 & 4 & Q21 & 1 & DYN-04 & 3 & Q22 & 2 & DYN-05 & 1 \\ \hline 
 
Q23 & 1 & DYN-06 & 0 & Q24 & 2 & TEC-01 & NT & Q24 & 1 & TEC-02 & 0 &  &  &  &  \\ \hline 
 
\end{tabular} 
\end{center} 
\normalsize 
 
\noindent \textbf{Bilan par compétences}
 
\begin{itemize} 
\item  \footnotesize \xpComp{CHS}{01} \normalsize \hspace{.2cm}Analyser un mécanisme en utilisant un graphe de liaisons\hfill \progress{100}
\item  \footnotesize \xpComp{CHS}{02} \normalsize \hspace{.2cm}Simplifier un mécanisme en utilisant une liaison équivalente\hfill \progress{100}
\item  \footnotesize \xpComp{CHS}{03} \normalsize \hspace{.2cm}Evaluer l'hyperstatisme d'un mécanisme\hfill \progress{0}
\item  \footnotesize \xpComp{CHS}{04} \normalsize \hspace{.2cm}Simplifier un mécanisme pour le rendre isostatique\hfill \progress{0}
\item  \footnotesize \xpComp{CHS}{05} \normalsize \hspace{.2cm}Analyser les conséquences de l'hyperstatisme d'un mécanisme\hfill \progress{0}
\item  \footnotesize \xpComp{CIN}{01} \normalsize \hspace{.2cm}Analyser un mécanisme, réaliser un graphe de liaison\hfill \progress{0}
\item  \footnotesize \xpComp{CIN}{02} \normalsize \hspace{.2cm}Déterminer un vecteur vitesse, un torseur cinématique, un vecteur accélération\hfill \progress{0}
\item  \footnotesize \xpComp{CIN}{03} \normalsize \hspace{.2cm}Déterminer le rapport de transmission d'un transmetteur\hfill \progress{60}
\item  \footnotesize \xpComp{CIN}{04} \normalsize \hspace{.2cm}Déterminer un loi ES cinématique, utiliser l'hypothèse de RSG\hfill \progress{40}
\item  \footnotesize \xpComp{CIN}{05} \normalsize \hspace{.2cm}Evaluer expérimentalement une grandeur cinématique\hfill \progress{100}
\item  \footnotesize \xpComp{DYN}{01} \normalsize \hspace{.2cm}Analyser un problème, définir une loi de mouvement \hfill \progress{0}
\item  \footnotesize \xpComp{DYN}{02} \normalsize \hspace{.2cm}Analyser un mécanisme en utilisant un graphe de structure\hfill \progress{20}
\item  \footnotesize \xpComp{DYN}{03} \normalsize \hspace{.2cm}Modéliser un solide et déterminer ses caractéristiques inertielles\hfill \progress{80}
\item  \footnotesize \xpComp{DYN}{04} \normalsize \hspace{.2cm}Déterminer un torseur cinétique, un torseur dynamique\hfill \progress{60}
\item  \footnotesize \xpComp{DYN}{05} \normalsize \hspace{.2cm}Proposer une démarche de résolution en utilisant le PFD\hfill \progress{20}
\item  \footnotesize \xpComp{DYN}{06} \normalsize \hspace{.2cm}Mettre en œuvre une démarche de résolution en utilisant le PFD\hfill \progress{0}
\item  \footnotesize \xpComp{GEO}{01} \normalsize \hspace{.2cm}Analyser la géométrie d'un mécanisme, analyser des surfaces de contact, réaliser des constructions géométriques\hfill \progress{70}
\item  \footnotesize \xpComp{GEO}{02} \normalsize \hspace{.2cm}Modéliser un mécanisme en réalisant un schéma cinématique paramétré\hfill \progress{0}
\item  \footnotesize \xpComp{GEO}{03} \normalsize \hspace{.2cm}Résoudre un problème de géométrie : déterminer la trajectoire d'un point ou déterminer une loi Entrée - Sortie\hfill \progress{0}
\item  \footnotesize \xpComp{GEO}{04} \normalsize \hspace{.2cm}Evaluer expérimentalement des grandeurs géométriques\hfill \progress{40}
\item  \footnotesize \xpComp{STAT}{01} \normalsize \hspace{.2cm}Analyser un problème en utilisant un graphe de structure\hfill \progress{80}
\item  \footnotesize \xpComp{STAT}{02} \normalsize \hspace{.2cm}Modéliser les actions mécaniques locales, globales, frottement\hfill \progress{0}
\item  \footnotesize \xpComp{STAT}{03} \normalsize \hspace{.2cm}Proposer une démarche de résolution en utilisant le PFS\hfill \progress{0}
\item  \footnotesize \xpComp{STAT}{04} \normalsize \hspace{.2cm}Mettre en œuvre une démarche de résolution\hfill \progress{0}
\item  \footnotesize \xpComp{STAT}{05} \normalsize \hspace{.2cm}Evaluer expérimentalement une action mécanique\hfill \progress{100}
\item  \footnotesize \xpComp{TEC}{01} \normalsize \hspace{.2cm}Analyser un mécanisme en utilisant un graphe de structure\hfill \progress{0}
\item  \footnotesize \xpComp{TEC}{02} \normalsize \hspace{.2cm}Déterminer les puissances intérieures\hfill \progress{0}
\end{itemize} 
\newpage 
\textbf{Bilan de compétences} 

\begin{minipage}[c]{.3\linewidth} 
\begin{tikzpicture}[scale=1, label distance = .5cm]
    \tkzKiviatDiagram[lattice=5,gap=.3,space=.3,step=1,radial style/.style ={-latex},lattice style/.style ={blue!30}]{SYS-01,SYS-02,SYS-03,SYS-04,SYS-05,SYS-06}
    \tkzKiviatLine[thick,color=red,mark=ball,ball color=red,mark size=3pt,fill=red!20](0.0,0.0,0.0,0.0,1.0000000000000002,4.0)
    \tkzKiviatLine[very thick, dotted, color=blue,mark=ball,mark size=1pt,fill=blue!20,	opacity=.5](1.857142857142857,1.7500000000000004,0.7142857142857144,1.25,1.8809523809523807,1.4047619047619047)
\end{tikzpicture} 
\end{minipage}\hfill 
\begin{minipage}[c]{.6\linewidth} 
\footnotesize 
\allSysComp
\normalsize 
\end{minipage} 
\begin{minipage}[c]{.3\linewidth} 
\begin{tikzpicture}[scale=1, label distance = .5cm]
    \tkzKiviatDiagram[lattice=5,gap=.3,space=.3,step=1,radial style/.style ={-latex},lattice style/.style ={blue!30}]{GEO-01,GEO-02,GEO-03,GEO-04}
    \tkzKiviatLine[thick,color=red,mark=ball,ball color=red,mark size=3pt,fill=red!20](3.1666666666666665,0.6000000000000001,0.0,1.0)
    \tkzKiviatLine[very thick, dotted, color=blue,mark=ball,mark size=1pt,fill=blue!20,	opacity=.5](1.2110317460317455,1.1666666666666667,0.8333333333333335,0.9464285714285713)
\end{tikzpicture} 
\end{minipage}\hfill 
\begin{minipage}[c]{.6\linewidth} 
\footnotesize 
\allGeoComp
\normalsize 
\end{minipage} 
\begin{minipage}[c]{.3\linewidth} 
\begin{tikzpicture}[scale=1, label distance = .5cm]
    \tkzKiviatDiagram[lattice=5,gap=.3,space=.3,step=1,radial style/.style ={-latex},lattice style/.style ={blue!30}]{CIN-01,CIN-02,CIN-03,CIN-04,CIN-05}
    \tkzKiviatLine[thick,color=red,mark=ball,ball color=red,mark size=3pt,fill=red!20](0.7,0.0,0.75,2.0000000000000004,1.6666666666666665)
    \tkzKiviatLine[very thick, dotted, color=blue,mark=ball,mark size=1pt,fill=blue!20,	opacity=.5](1.055190476190476,1.2023809523809528,1.2232142857142863,1.9826984126984124,1.654761904761905)
\end{tikzpicture} 
\end{minipage}\hfill 
\begin{minipage}[c]{.6\linewidth} 
\footnotesize 
\allCinComp
\normalsize 
\end{minipage} 
\begin{minipage}[c]{.3\linewidth} 
\begin{tikzpicture}[scale=1, label distance = .5cm]
    \tkzKiviatDiagram[lattice=5,gap=.3,space=.3,step=1,radial style/.style ={-latex},lattice style/.style ={blue!30}]{STAT-01,STAT-02,STAT-03,STAT-04,STAT-05}
    \tkzKiviatLine[thick,color=red,mark=ball,ball color=red,mark size=3pt,fill=red!20](1.6,1.5,1.0,3.75,4.666666666666666)
    \tkzKiviatLine[very thick, dotted, color=blue,mark=ball,mark size=1pt,fill=blue!20,	opacity=.5](1.926190476190476,1.6964285714285716,1.9423809523809528,2.4732142857142856,3.1626984126984126)
\end{tikzpicture} 
\end{minipage}\hfill 
\begin{minipage}[c]{.6\linewidth} 
\footnotesize 
\allStatComp
\normalsize 
\end{minipage} 
\begin{minipage}[c]{.3\linewidth} 
\begin{tikzpicture}[scale=1, label distance = .5cm]
    \tkzKiviatDiagram[lattice=5,gap=.3,space=.3,step=1,radial style/.style ={-latex},lattice style/.style ={blue!30}]{CHS-01,CHS-02,CHS-03,CHS-04,CHS-05}
    \tkzKiviatLine[thick,color=red,mark=ball,ball color=red,mark size=3pt,fill=red!20](1.6666666666666665,2.0,0.0,2.5,3.75)
    \tkzKiviatLine[very thick, dotted, color=blue,mark=ball,mark size=1pt,fill=blue!20,	opacity=.5](2.6944444444444446,2.0142857142857147,2.500000000000001,2.440476190476192,2.383928571428572)
\end{tikzpicture} 
\end{minipage}\hfill 
\begin{minipage}[c]{.6\linewidth} 
\footnotesize 
\allChsComp
\normalsize 
\end{minipage} 
\begin{minipage}[c]{.3\linewidth} 
\begin{tikzpicture}[scale=1, label distance = .5cm]
    \tkzKiviatDiagram[lattice=5,gap=.3,space=.3,step=1,radial style/.style ={-latex},lattice style/.style ={blue!30}]{DYN-01,DYN-02,DYN-03,DYN-04,DYN-05,DYN-06}
    \tkzKiviatLine[thick,color=red,mark=ball,ball color=red,mark size=3pt,fill=red!20](1.6666666666666665,0.3333333333333333,1.6,1.5,0.5,0.7500000000000001)
    \tkzKiviatLine[very thick, dotted, color=blue,mark=ball,mark size=1pt,fill=blue!20,	opacity=.5](1.630952380952381,1.8650793650793651,1.6119047619047622,0.6071428571428572,1.1488095238095237,1.4880952380952381)
\end{tikzpicture} 
\end{minipage}\hfill 
\begin{minipage}[c]{.6\linewidth} 
\footnotesize 
\allDynComp
\normalsize 
\end{minipage} 
\begin{minipage}[c]{.3\linewidth} 
\begin{tikzpicture}[scale=1, label distance = .5cm]
    \tkzKiviatDiagram[lattice=5,gap=.3,space=.3,step=1,radial style/.style ={-latex},lattice style/.style ={blue!30}]{TEC-01,TEC-02,TEC-03,TEC-04,TEC-05}
    \tkzKiviatLine[thick,color=red,mark=ball,ball color=red,mark size=3pt,fill=red!20](2.2,0.9999999999999999,0.7500000000000001,1.3333333333333333,0.0)
    \tkzKiviatLine[very thick, dotted, color=blue,mark=ball,mark size=1pt,fill=blue!20,	opacity=.5](1.4103809523809518,0.9107142857142859,1.1249999999999998,1.2380952380952377,0.8904761904761905)
\end{tikzpicture} 
\end{minipage}\hfill 
\begin{minipage}[c]{.6\linewidth} 
\footnotesize 
\allTecComp
\normalsize 
\end{minipage} 
\begin{minipage}[c]{.3\linewidth} 
\begin{tikzpicture}[scale=1, label distance = .5cm]
    \tkzKiviatDiagram[lattice=5,gap=.3,space=.3,step=1,radial style/.style ={-latex},lattice style/.style ={blue!30}]{SLCI-01,SLCI-02,SLCI-03,SLCI-04,SLCI-05,SLCI-06,SLCI-07,SLCI-08,SLCI-09,SLCI-10,SLCI-11}
    \tkzKiviatLine[thick,color=red,mark=ball,ball color=red,mark size=3pt,fill=red!20](1.4999999999999996,1.5,1.5000000000000002,2.75,2.0,0.0,2.25,1.0,5.0,4.25,2.5)
    \tkzKiviatLine[very thick, dotted, color=blue,mark=ball,mark size=1pt,fill=blue!20,	opacity=.5](1.4635034013605437,1.083333333333333,1.931428571428572,1.8392857142857144,2.291666666666666,2.1934523809523805,1.056547619047619,2.6328571428571435,2.9107142857142856,3.154761904761904,2.4821428571428577)
\end{tikzpicture} 
\end{minipage}\hfill 
\begin{minipage}[c]{.6\linewidth} 
\footnotesize 
\allSlciComp
\normalsize 
\end{minipage} 
\begin{minipage}[c]{.3\linewidth} 
\begin{tikzpicture}[scale=1, label distance = .5cm]
    \tkzKiviatDiagram[lattice=5,gap=.3,space=.3,step=1,radial style/.style ={-latex},lattice style/.style ={blue!30}]{PERF-01,PERF-02,PERF-03,PERF-04,PERF-05,PERF-06}
    \tkzKiviatLine[thick,color=red,mark=ball,ball color=red,mark size=3pt,fill=red!20](0.0,0.0,3.0,1.25,3.666666666666667,1.6)
    \tkzKiviatLine[very thick, dotted, color=blue,mark=ball,mark size=1pt,fill=blue!20,	opacity=.5](1.8333333333333335,2.678571428571429,2.164285714285716,1.6726190476190474,2.2222222222222223,1.6761904761904767)
\end{tikzpicture} 
\end{minipage}\hfill 
\begin{minipage}[c]{.6\linewidth} 
\footnotesize 
\allPerfComp
\normalsize 
\end{minipage} 
\begin{minipage}[c]{.3\linewidth} 
\begin{tikzpicture}[scale=1, label distance = .5cm]
    \tkzKiviatDiagram[lattice=5,gap=.3,space=.3,step=1,radial style/.style ={-latex},lattice style/.style ={blue!30}]{COR-01,COR-02,COR-03,COR-04,COR-05,COR-06}
    \tkzKiviatLine[thick,color=red,mark=ball,ball color=red,mark size=3pt,fill=red!20](2.5,0.2,0.0,0.6000000000000001,1.6,1.5)
    \tkzKiviatLine[very thick, dotted, color=blue,mark=ball,mark size=1pt,fill=blue!20,	opacity=.5](1.1766190476190475,1.0333333333333334,1.055059523809524,0.830952380952381,1.1904761904761905,0.7797619047619044)
\end{tikzpicture} 
\end{minipage}\hfill 
\begin{minipage}[c]{.6\linewidth} 
\footnotesize 
\allCorComp
\normalsize 
\end{minipage} 
\begin{minipage}[c]{.3\linewidth} 
\begin{tikzpicture}[scale=1, label distance = .5cm]
    \tkzKiviatDiagram[lattice=5,gap=.3,space=.3,step=1,radial style/.style ={-latex},lattice style/.style ={blue!30}]{NL-01,NL-02}
    \tkzKiviatLine[thick,color=red,mark=ball,ball color=red,mark size=3pt,fill=red!20](2.0,1.0)
    \tkzKiviatLine[very thick, dotted, color=blue,mark=ball,mark size=1pt,fill=blue!20,	opacity=.5](0.9345238095238094,1.2161904761904763)
\end{tikzpicture} 
\end{minipage}\hfill 
\begin{minipage}[c]{.6\linewidth} 
\footnotesize 
\allNlComp
\normalsize 
\end{minipage} 
\begin{minipage}[c]{.3\linewidth} 
\begin{tikzpicture}[scale=1, label distance = .5cm]
    \tkzKiviatDiagram[lattice=5,gap=.3,space=.3,step=1,radial style/.style ={-latex},lattice style/.style ={blue!30}]{SEQ-01,SEQ-02,SEQ-03}
    \tkzKiviatLine[thick,color=red,mark=ball,ball color=red,mark size=3pt,fill=red!20](2.5,2.0,0.0)
    \tkzKiviatLine[very thick, dotted, color=blue,mark=ball,mark size=1pt,fill=blue!20,	opacity=.5](1.2202380952380951,1.8273809523809528,2.5714285714285716)
\end{tikzpicture} 
\end{minipage}\hfill 
\begin{minipage}[c]{.6\linewidth} 
\footnotesize 
\allSeqComp
\normalsize 
\end{minipage} 
\begin{minipage}[c]{.3\linewidth} 
\begin{tikzpicture}[scale=1, label distance = .5cm]
    \tkzKiviatDiagram[lattice=5,gap=.3,space=.3,step=1,radial style/.style ={-latex},lattice style/.style ={blue!30}]{NUM-01,NUM-02,NUM-03,NUM-04,NUM-05}
    \tkzKiviatLine[thick,color=red,mark=ball,ball color=red,mark size=3pt,fill=red!20](0.0,0.0,5.0,5.0,5.0)
    \tkzKiviatLine[very thick, dotted, color=blue,mark=ball,mark size=1pt,fill=blue!20,	opacity=.5](1.7619047619047619,3.142857142857143,3.1309523809523823,2.9404761904761907,2.392857142857143)
\end{tikzpicture} 
\end{minipage}\hfill 
\begin{minipage}[c]{.6\linewidth} 
\footnotesize 
\allNumComp
\normalsize 
\end{minipage} 
