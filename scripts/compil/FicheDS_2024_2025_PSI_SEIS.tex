\documentclass[10pt,fleqn]{book} % Default font size and left-justified equations
\usepackage[%
    pdftitle={Modélisation systèmes multiphysiques : Modélisation par fonction de transfert et schéma-blocs},
    pdfauthor={Xavier Pessoles}]{hyperref}
\input{style/new_style}
%\input{../../../Style/macros_SII}
\input{../../../Style/environment_v2}
\def\classe{\textsf{PSI$\star$}}
\def\xxnumpartie{}
\def\xxpartie{}%Modéliser et corriger le comportement linéaire et non linéaire des systèmes multiphysiques}
\def\xxnumchapitre{Devoir surveillé de SII 3\vspace{.2cm}}
\def\xxchapitre{\hspace{.12cm}  }
\def\discipline{Sciences \\Industrielles de \\ l'Ingénieur}
\def\xxtete{Sciences Industrielles de l'Ingénieur}
\def\xxactivite{\textsf{DS 3}}
\def\xxonglet{\textsf{DS 3}}
\def\xxauteur{\textsl{Xavier Pessoles}}

\def\xxtitreexo{\noindent Table Azalée}
\def\xxsourceexo{\hspace{.2cm} Banque PT -- SIA 2019}


\def\xxfigures{\includegraphics[width=.6\linewidth]{fig_00}}
\def\xxpied{%
\textsf{DS 3}%dans le but de déterminer les contraintes géométriques dans les mécanismes\\% afin de valider leurs performances.\\
%Révisions 1 -- 2 -- 3 -- \xxactivite%
}

\def\xxposongletx{2}
\def\xxposonglettext{1.45}
\def\xxposonglety{20}
%\def\xxonglet{Part. 1 -- Ch. 3}


\usepackage{multicol}
\usepackage{siunitx}

\usepackage{tkz-kiviat}
\fichetrue
%\fichefalse


%\newcommand{\progress}[1]{
%    \begin{tikzpicture}
%        \fill[ocre!20, rounded corners =1 mm] (0,0) rectangle (#1/100*2.5,.4);
%         \draw[ocre,thick, rounded corners =1 mm] (0,0) rectangle (2.5,.4);
%         \node at (1.25,0.2) {\textcolor{black}{#1~\%}};
%    \end{tikzpicture}
%}

\proftrue
\proffalse

\tdtrue
\tdfalse

\courstrue
\coursfalse

\setcounter{secnumdepth}{5}
%---------------------------------------------------------------------------


\begin{document}
%\chapterimage{png/Fond_Cin}
%%\renewcommand{\xxactivite}{DS 1}
\renewcommand{\xxtitreexo}{Devoir du 30 septembre 2024}
%\renewcommand{\discipline}{Sciences Industrielles de l'Ingénieur}
%\renewcommand{\classe}{PSI $\star$}
%\renewcommand{\xxnumpartie}{PSI $\star$}
%\renewcommand{\xxsourceexo}{PSI $\star$}
%\renewcommand{\xxchapitre}{PSI $\star$}
%\renewcommand{\xxnumpartie}{PSI $\star$}	
%\renewcommand{\xxpartie}{PSI $\star$}

\input{style/new_pagegarde}
%\vspace{4.5cm}
\pagestyle{fancy}
\thispagestyle{plain}


\def\columnseprulecolor{\color{ocre}}
\setlength{\columnseprule}{0pt} 


\begin{tikzpicture}[remember picture,overlay]
\draw[anchor=west] (-2cm,-4cm) node {\huge\sffamily\bfseries\color{black} %
\begin{minipage}{5cm}
\begin{center}
\LARGE\sffamily\color{ocre}\textbf{\textsc{\xxactivite}}

\begin{center}
\xxfigures
\end{center}

\end{center}
\end{minipage} \hfill
\begin{minipage}[c]{12cm}
\begin{titrechapitre}
\renewcommand{\baselinestretch}{1.1} 
\large\sffamily\textbf{\textsc{\xxtitreexo}}

\small\sffamily{\textbf{\textit{\color{black!70}\xxsourceexo}}}
\vspace{.5cm}

\renewcommand{\baselinestretch}{1} 
\normalsize\normalfont
%\xxcompetences
\end{titrechapitre}
\end{minipage}};
\end{tikzpicture}

\vspace{6cm}
%

\textbf{Remarques générales}
\begin{itemize}
\item Lorsqu'on isole deux solides, il est inutile (et une perte de temps) d'exprimer le torseur de la pesanteur au barycentre des solides. Vous pouvez diretement le déplacer (si nécessaire) au point d'expression du TMS.
\item Le PFS, en SI, est une équation entre torseur. Le TMS (théorème du moment statique) est une équation vectorielle. Le TRS (théorème de la résultante statique) est une équation vectorielle. Attention à l'utilisation abusive de l'acronyme PFD (notamment lorsqu'on est en statique) ou de l'acronyme TMC. 
\item Belle prestation sur le calcul du rapport du train épicycloïdal !
\item Un peu plus d'application est souhaitable sur le schéma cinématique. 
\end{itemize}
%
%\textbf{Remarques particulières}
%\begin{itemize}
%\item Des confusions entre stabilité, notamment à la question 19. Attention, la question portait sur la stabilité. (De plus la précision ne peut être déterminée que si le système est stable).
%\item Pour la stabilité, un système en BF \textbf{d'ordre 1 ou d'ordre 2} est stable si les coefficients sont tous \textbf{strictement} positifs. Pour des ordres supérieurs, cette condition n'est pas pas suffisante (mais elle est nécessaire).
%\item Attention à utiliser des unités adaptées. On ne va pas exprimer une vitesse en \si{W.N^{-1}}...
%\item Attention à calculer l'écart statique que quand c'est nécessaire, sinon utiliser le tableau du cours (en liaison avec la classe de la BO).
%\item Pour calculer l'argument d'une fonction de transfert, si on a des polynômes de degré 1, c'est génial, car on n'est dans le bon domaine de définition de l'arctan... Donc ne développez pas les formes factorisées...
%\item Attention à calculer les fonctions de transfert demandées... et pas celle que vous voudriez qu'on vous demande.  
%\end{itemize}

\begin{minipage}[c]{.45\linewidth} 
\Large \textbf{\textsf{NOM19 Prenom19}} 
 
 \normalsize Note harmonisée 6.11/20 
 
Rang 6
 
Moyenne classe harmonisée 7.13/20 
 
Moyenne question traitées 6.82/20 
 
Rang question traitées 3 
 
Commentaires : 
Comment20 
\end{minipage}\hfill 
\begin{minipage}[c]{.45\linewidth}  
\begin{center}
\includegraphics[width=.8\linewidth]{../histo.pdf} 
\end{center}
\end{minipage}
\footnotesize 
\begin{center} 
\begin{tabular}{|c|c|m{1cm}|c||c|c|m{1cm}|c||c|c|m{1cm}|c||c|c|m{1cm}|c|} 
\hline \textbf{Qu} & \textbf{Coef} & \textbf{Comp} & \textbf{/5} & \textbf{Qu} & \textbf{Coef} & \textbf{Comp} & \textbf{/5} & \textbf{Qu} & \textbf{Coef} & \textbf{Comp} & \textbf{/5} & \textbf{Qu} & \textbf{Coef} & \textbf{Comp} & \textbf{/5} \\ 
\hline 
\hline 
Q1 & 5 & A1-01 & NT & Q1 & 5 & A1-02 & 2 & Q2 & 5 & A1-03 & 5 & Q2 & 5 & A1-04 & 0 \\ \hline 
 
Q3 & 5 & A1-05 & 1 & Q4 & 5 & A2-01 & 4 & Q5 & 5 & A2-02 & NT & Q6 & 5 & A2-03 & 2 \\ \hline 
 
Q7 & 5 & A3-01 & 5 & Q8 & 5 & A3-02 & 0 & Q9 & 10 & A3-03 & 1 &  &  &  &  \\ \hline 
 
\end{tabular} 
\end{center} 
\normalsize 
 






\end{document}
